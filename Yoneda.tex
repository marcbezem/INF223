\documentclass[handout]{beamer}
%\documentclass[slides]{beamer}
% Vary the color applet  (try out your own if you like)
%\colorlet{structure}{red!20!black}
%\beamertemplateshadingbackground{yellow!20}{white}
%\usepackage{beamerthemeshadow}
%\usepackage[utf8x]{inputenc} CONFLICT!
\usepackage{tikz}
%\usepackage[english,norsk,nynorsk]{babel}
\usepackage{tikz-cd}
\usetikzlibrary{trees}

\usepackage[all]{xy}
\usepackage{multicol}

%\setbeamertemplate{navigation symbols}{}++++++
%\setbeamertemplate{footline}[frame number]
\usetheme{Montpellier}


\input macros

\newcommand{\To}{\Rightarrow}
\newcommand{\Trt}{\stackrel{*}{\Rightarrow}}
\newcommand{\ToG}{\Rightarrow_G}
\newcommand{\redS}{{\color{red} S}}

\newcommand{\bfsf}[1]{{\boldsymbol{#1}}}
\newcommand{\Set}{\bfsf{Set}}
\newcommand{\Gra}{\bfsf{Graph}}
\newcommand{\CC}{\bfsf{C}}
\newcommand{\DD}{\bfsf{D}}
\newcommand{\EE}{\bfsf{E}}
\newcommand{\PP}{\bfsf{P}}
\newcommand{\HH}{\bfsf{H}}
\newcommand{\Nat}{\bfsf{Nat}}
\newcommand{\Incl}{\bfsf{Incl}}
\newcommand{\Rel}{\bfsf{Rel}}
\newcommand{\Mult}{\bfsf{Mult}}
\newcommand{\Mon}{\bfsf{Mon}}
\newcommand{\Cat}{\bfsf{Cat}}
\newcommand{\CAT}{\bfsf{CAT}}

\newcommand{\Kp}[1]{{\langle #1 \rangle}}
\newcommand{\Kc}{;\!;}
\newcommand{\bind}{{>}\!\!{>}\!{=}}
\newcommand{\ttt}[1]{\text{\tt #1}}

\title[INF223 presentations]{}

\begin{document}

\section{Monads}
\subsection{Introduction}
 
\frame
  {   
    \frametitle{You need a lemma}\label{Yon:Intro}

 \begin{itemize}[<+->]
\item Assume we have the following data:
\begin{itemize}
    \item a small category $\CC$
    \item a functor $F:\CC\to\Set$
    \item an object $c$ of $\CC$, so a functor $\CC(c,\_):\CC\to\Set$
 \end{itemize}
Then the set of natural transformations $\CC(c,\_)\To F$ is in bijective
correspondence with the set $F(c)$, and this in a `natural' way
(we will explain this and make this precise)
\item The above statement is called the Yoneda Lemma
\item Useful links are:
 \begin{itemize}
    \item \myurl{https://en.wikipedia.org/wiki/Yoneda\_lemma}
    \item \myurl{https://ncatlab.org/nlab/show/Yoneda+lemma}
 \end{itemize}
 \end{itemize}

 }

\frame
  {   
    \frametitle{Exploration of the simplest possible case}\label{Yon:Triv}

 \begin{itemize}[<+->]
\item Let $\CC$ be the trivial category, with only one object $1$ and
 its identity morphism $id_1$. Then $\CC(1,\_):\CC\to\Set$ maps
$1$ to $\CC(1,1) = \set{id_1}$ and $\CC(1,id_1) = id_{\set{id_1}}$
\item Let $F:\CC\to\Set$ be a functor, then $F(1)$ is a set, and
$F(id_1) = id_{F(1)}$
\item Any natural transformation $\alpha: \CC(c,\_)\To F$ consists of
only $\alpha_1: \set{id_1} \to F(1)$ and,
conversely, any such map is a natural transformation
since the only morphism in $\CC$ is $id_1$
\item Yoneda now says:  $(\set{id_1} \to F(1)) \simeq F(1)$ in $\Set$
\item Since $F$ was arbitrary we get  $(1_\Set \to X) \simeq X$ for every set $X$
\item Thus the Yoneda lemma is a (vast) generalization of this simple fact
 \end{itemize}

 }

\frame
  {   
    \frametitle{Exploration of the next simplest case}\label{Yon:Triv}

 \begin{itemize}[<+->]
\item Let $\CC$ be the category ${\cdot}{\to}{\cdot}$, with objects $1,2$ and
a non-trivial morphism $f:1\to 2$. Then $\CC(1,\_):\CC\to\Set$ maps
$1$ to $\CC(1,1) = \set{id_1}$ and $2$ to $\CC(1,2) = \set{f}$, and
$\CC(1,f)$ is the map $\CC(1,1)\to\CC(1,2)$ sending $id_1$ to $f$  
\item Let $F:\CC\to\Set$ be a functor, then $F(1), F(2)$ are sets, and
$F(f)$ is a map ${F(1)}\to\set{f}$
\item Any natural transformation $\alpha: \CC(1,\_)\To F$ consists of
$\alpha_1: \set{id_1} \to F(1)$ and $\alpha_2: \set{f} \to F(2)$ satisfying 
naturality: $F(f)(\alpha_1(id_1)) = \alpha_2(f)$, i.e., an element of the graph of $F(f)$
\item Notations: $X_i = F(i)$ and $x_i =\alpha_i(\_)$ with % $x_i \in X_i$ and
$F(f)(x_1) = x_2$
\item Yoneda now says:   $F(1) \simeq{}$ the graph of $F(f)$
\item This boils down to: domain and graph of $F(f)$ are isomorphic
\item Again  the Yoneda lemma generalizes (vastly) a simple fact
 \end{itemize}

 }

\frame
  {   
    \frametitle{Representable functors to $\Set$}\label{Yon:ReprFun}

 \begin{itemize}[<+->]
\item Let $\CC$ be a small category and $A$ an arbitrary object in $\CC$. 
The functor  $\CC(A,\_):\CC\to\Set$ maps objects $X$ to sets $\CC(A,X)$
of morphisms from $A$ to $X$. Morphisms $f: X\to Y$ in $\CC$ are
mapped to Morphisms $\CC(A,f): \CC(A,X)\to\CC(A,Y)$ by
postcompostion with $f$: if $g:A\to X$ then $\CC(A,f)(g) = g;f$.
\item A functor $F:\CC\to\Set$ is \emph{represented} by $A$ if
there exists a natural isomorphism $\alpha: \CC(A,\_)\To F$.
Intuition: $F$ describes the social life of $A$ `on the right'.
\item Dually, we have a functor $\CC(\_,A):\CC^{op}\to\Set$
and a notion of corepresentable functor, 
describing the social life of $A$ `on the left' 
(both `left' and `right' include $A$ itself).
\item All (co)representable functors together describe $\CC$ somehow
\item NB Yoneda gives information on natural transformations 
$\alpha: \CC(A,\_)\To F$, not necessarily being natural isomorphisms

\end{itemize}

 }

\frame
  {   
    \frametitle{Exploration of the monoid case}\label{Yon:Mon}

 \begin{itemize}[<+->]
\item Let $\CC$ be a monoid $(M,1,*)$ as a category, i.e., $\CC$ has one object
$\bullet$ and morphisms $m:\bullet\to\bullet$ for 
every $m\in M$, with composition and unit as in the monoid.
The functor $\CC(\bullet,\_)$ acting on $m$ is right multiplication with $m$ (why?).
\item A functor $F:\CC\to\Set$ consist of a set $F(\bullet)$ and for every
$m\in M$ a map $F(m) : F(\bullet)\to F(\bullet)$ such that $F(1)= id_{F(\bullet)}$
and $F(m*m')=F(m);F(m')$.
\item Any natural transformation $\alpha: \CC(\bullet,\_)\To F$ consists of
a set 

\item Yoneda now says:   $F(1) \simeq{}$ the graph of $F(f)$
\item This boils down to: domain and graph of $F(f)$ are isomorphic
\item Again  the Yoneda lemma generalizes (vastly) a simple fact
 \end{itemize}

 }


\end{document}

\myurl{en.wikipedia.org/wiki/Ordered_pair}
 \begin{itemize}
    \item emulation halts because of the partiality of $\delta$;
    \item emulation halts because of "head left" with head at first cell;
    \item emulation halts because of reaching a halting state;
    \item the emulated TM goes on forever.
 \end{itemize}
