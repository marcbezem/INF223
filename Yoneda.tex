\documentclass[handout]{beamer}
%\documentclass[slides]{beamer}
% Vary the color applet  (try out your own if you like)
%\colorlet{structure}{red!20!black}
%\beamertemplateshadingbackground{yellow!20}{white}
%\usepackage{beamerthemeshadow}
%\usepackage[utf8x]{inputenc} CONFLICT!
\usepackage{tikz}
%\usepackage[english,norsk,nynorsk]{babel}
\usepackage{tikz-cd}
\usetikzlibrary{trees}

\usepackage[all]{xy}
\usepackage{multicol}

%\setbeamertemplate{navigation symbols}{}++++++
%\setbeamertemplate{footline}[frame number]
\usetheme{Montpellier}


\input macros

\newcommand{\To}{\Rightarrow}
\newcommand{\Trt}{\stackrel{*}{\Rightarrow}}
\newcommand{\ToG}{\Rightarrow_G}
\newcommand{\redc}{{\color{red} c}}


\newcommand{\bfsf}[1]{{\boldsymbol{#1}}}
\newcommand{\Set}{\bfsf{Set}}
\newcommand{\Gra}{\bfsf{Graph}}
\newcommand{\CC}{\bfsf{C}}
\newcommand{\DD}{\bfsf{D}}
\newcommand{\EE}{\bfsf{E}}
\newcommand{\PP}{\bfsf{P}}
\newcommand{\HH}{\bfsf{H}}
\newcommand{\Nat}{\bfsf{Nat}}
\newcommand{\Incl}{\bfsf{Incl}}
\newcommand{\Rel}{\bfsf{Rel}}
\newcommand{\Mult}{\bfsf{Mult}}
\newcommand{\Mon}{\bfsf{Mon}}
\newcommand{\Cat}{\bfsf{Cat}}
\newcommand{\CAT}{\bfsf{CAT}}

\newcommand{\Kp}[1]{{\langle #1 \rangle}}
\newcommand{\Kc}{;\!;}
\newcommand{\bind}{{>}\!\!{>}\!{=}}
\newcommand{\ttt}[1]{\text{\tt #1}}

\title[INF223 presentations]{}

\begin{document}

\section{Monads}
\subsection{Introduction}
 
\frame
  {   
    \frametitle{The Yoneda Lemma}\label{Yon:Intro}

 \begin{itemize}[<+->]
\item Assume we have the following data:
\begin{itemize}
    \item a small category $\CC$
    \item a functor $F:\CC\to\Set$
    \item an object $c$ of $\CC$, so a functor $\CC(c,\_):\CC\to\Set$
 \end{itemize}
Then the set of natural transformations $\CC(c,\_)\To F$ is in bijective
correspondence with the set $F(c)$, and this in a `natural' way
(we will explain this and make this precise)
\item The above statement is called the 
\href{https://en.wikipedia.org/wiki/Nobuo_Yoneda}{\color{blue}Yoneda} Lemma
\item Definition of the bijection is a sequence of forced moves: given $\alpha: \CC(c,\_)\To F$
construct an element of  $F(c)$ in the only way that always works (even if $\CC$ is trivial,
{\color{red}try}!)
\item Useful links are:
 \begin{itemize}
    \item \myurl{https://en.wikipedia.org/wiki/Yoneda\_lemma}
    \item \myurl{https://ncatlab.org/nlab/show/Yoneda+lemma}
 \end{itemize}
 \end{itemize}

 }

\frame
  {   
    \frametitle{Exploration of the simplest possible case}\label{Yon:Triv}

 \begin{itemize}[<+->]
\item Let $\CC$ be the trivial category, with only one object $1$ and
 its identity morphism $id_1$. Then $\CC(1,\_):\CC\to\Set$ maps
$1$ to $\CC(1,1) = \set{id_1}$ and $\CC(1,id_1) = id_{\set{id_1}}$
\item Let $F:\CC\to\Set$ be a functor, then $F(1)$ is a set, and
$F(id_1) = id_{F(1)}$
\item Any natural transformation $\alpha: \CC(1,\_)\To F$ consists of
only $\alpha_1: \set{id_1} \to F(1)$ and,
conversely, any such map is a natural transformation
since the only morphism in $\CC$ is $id_1$
\item Yoneda now says:  $(\set{id_1} \to F(1)) \simeq F(1)$ in $\Set$
\item Since $F$ was arbitrary we get  $(1_\Set \to X) \simeq X$ for every set $X$
\item Thus the Yoneda lemma is a (vast) generalization of this simple fact
 \end{itemize}

 }

\frame
  {   
    \frametitle{Exploration of the next simplest case}\label{Yon:Triv}

 \begin{itemize}[<+->]
\item Let $\CC$ be the category ${\cdot}{\to}{\cdot}$, with objects $1,2$ and
one morphism $f:1\to 2$, besides $id_1,id_2$. Then $\CC(1,\_):\CC\to\Set$ maps
$1$ to $\CC(1,1) = \set{id_1}$ and $2$ to $\CC(1,2) = \set{f}$, and
$\CC(1,f)$ is the map $\CC(1,1)\to\CC(1,2)$ sending $id_1$ to $f$  
\item Let $F:\CC\to\Set$ be a functor, then $F(1), F(2)$ are sets, and
$F(f)$ is a map ${F(1)}\to{F(2)}$
\item Any natural transformation $\alpha: \CC(1,\_)\To F$ consists of
$\alpha_1: \set{id_1} \to F(1)$ and $\alpha_2: \set{f} \to F(2)$ satisfying 
naturality: $F(f)(\alpha_1(id_1)) = \alpha_2(f)$, i.e., an element of the graph of $F(f)$
\item Conversely, if $F(f)(x_1) = x_2$ then  $\alpha_i(\_) = x_i$ is natural

\item Yoneda now implies:   $F(1) \simeq{}$ the graph of $F(f)$
\item Generally: domain and graph of any function are isomorphic
\item Again  the Yoneda lemma generalizes (vastly) a simple fact
 \end{itemize}

 }

\frame
  {   
    \frametitle{Representable functors to $\Set$}\label{Yon:ReprFun}

 \begin{itemize}[<+->]
\item Let $\CC$ be a small category and $A$ an arbitrary object in $\CC$. 
The functor  $\CC(A,\_):\CC\to\Set$ maps objects $X$ to sets $\CC(A,X)$
of morphisms from $A$ to $X$. Morphisms $f: X\to Y$ in $\CC$ are
mapped to morphisms $\CC(A,f): \CC(A,X)\to\CC(A,Y)$ by
postcompostion with $f$: if $g:A\to X$ then $\CC(A,f)(g) = g;f$.
\item A functor $F:\CC\to\Set$ is \emph{represented} by $A$ if
there exists a natural isomorphism $\alpha: \CC(A,\_)\To F$.
Intuition: $F$ describes the social life of $A$ `on the right'.
\item Dually, we have a functor $\CC(\_,A):\CC^{op}\to\Set$
and a notion of corepresentable functor, 
describing the social life of $A$ `on the left' 
(both `left' and `right' include $A$ itself).
\item All (co)representable functors together describe $\CC$ `somehow'
\item Later: the Yoneda embedding makes this more precise

\end{itemize}

 }

\frame
  {   
    \frametitle{Exploration of the monoid case}\label{Yon:Mon}

 \begin{itemize}[<+->]
\item Let $\CC$ be a monoid $(M,1,*)$ as a category, i.e., $\CC$ has one object
$\bullet$ and $\CC(\bullet,\bullet)=M$, with composition and identity as in the monoid.
The map $\CC(\bullet,m)$ is $x\mapsto x*m$ for each $m\in M$ %(why?).
\item A functor $F:\CC\to\Set$ consist of a set $F(\bullet)$ plus for every
$m\in M$ a map $F(m) : F(\bullet)\to F(\bullet)$ such that $F(1)= id_{F(\bullet)}$
and $F(m*m')=F(m);F(m')$. In fact, $F$ is a monoid homomorphism from
$(M,1,*)$ to the \emph{transformation monoid} of $F(\bullet)$, that is,
$(F(\bullet)\to F(\bullet), id_{F(\bullet)}, ;)$.
\item Any natural transformation $\alpha: \CC(\bullet,\_)\To F$ is
a map $\CC(\bullet,\bullet)\to F(\bullet)$, so $M\to F(\bullet)$ satisfying the naturality square
\begin{tikzcd}[ampersand replacement=\&]
\bullet \arrow[swap]{d}{m} \& M  \arrow[swap]{d}{\_*m}\arrow{r}{\alpha} \&\arrow{d}{F(m)} F(\bullet)\\
\bullet \& M \arrow{r}{\alpha} \& F(\bullet)
\end{tikzcd}
$\begin{array}{l}%\vspace*{2mm}
\text{implying that for all $x,m\in M$}\\
\text{$F(m)(\alpha(x)) = \alpha(x*m)$, so}\\
\text{$\alpha(m) = F(m)(\alpha(1))$, i.e.\ $\alpha(1)$}\\
\text{in $F(\bullet)$ determines $\alpha$: Yoneda}
\end{array}$

 \end{itemize}

 }

\frame
  {   
    \frametitle{Corollary of the monoid case: Cayley's Theorem}\label{Yon:MonCayley}

 \begin{itemize}[<+->]
\item Let $\CC$ be as before and take for $F$ also $\CC(\bullet,\_)$
\item Any natural transformation $\alpha: \CC(\bullet,\_)\To \CC(\bullet,\_)$ is
then a map $M\to M$ satisfying $\alpha(x)*m = \alpha(x*m)$,
\begin{tikzcd}[ampersand replacement=\&]
\bullet \arrow[swap]{d}{m} \& M  \arrow[swap]{d}{\_*m}\arrow{r}{\alpha} \&\arrow{d}{\_*m} M\\
\bullet \& M \arrow{r}{\alpha} \& M
\end{tikzcd}
$\begin{array}{l}%\vspace*{2mm}
\text{completely determined by $\alpha(1)$:}\\
\text{$\alpha(m) = \alpha(1*m)   = \alpha(1)*m$,}\\
\text{so $\alpha = (m\mapsto \alpha(1)*m)$, and}\\ %= (\alpha(1)*\_)
\text{$\alpha(1)$ can be any element in $M$}
\end{array}$
\item Mapping such $\alpha$'s to $\alpha(1)\in M$ is a bijection, with inverse
mapping each $n\in M$ to $\alpha_n = (m\mapsto n*m)$ %, the \emph{left action} of $n$ on $M$
\item Note that $\alpha_1 = id_M$ and 
$\alpha_{n*n'} = \alpha_{n'};\alpha_n= \alpha_n\circ\alpha_{n'}$, so
$n\mapsto\alpha_n$ defines a monoid isomorphism  from $(M,1,*)$
to a submonoid of $M$'s transformation monoid: Cayley's Theorem
\item Again  the Yoneda lemma generalizes a known fact
 \end{itemize}

}

\frame
  {   
    \frametitle{You need a lemma, so formulate and prove it!}\label{Yon:Intro}

 \begin{itemize}[<+->]
\item Given a small category $\CC$, a functor $F:\CC\to\Set$ and an
object $c$ of $\CC$, then the map $\Phi_c(\alpha) = \alpha_c(id_c)$ is
a bijection $(\CC(c,\_)\To F)\to F(c)$ with inverse mapping $x$ to $\beta$
with $\beta_d(g)=F(g)(x)$ for any object $d$ of $\CC$ and $g\in\CC(c,d)$.
These bijections are natural in object $c$ and functor $F$.
\item Proving the Yoneda Lemma means taking the following steps:
 \begin{enumerate}
    \item Prove that $x\mapsto\beta$ is a well-defined inverse of $\Phi_c$
    \item Extend $c\mapsto(\CC(c,\_)\To F)$ to a functor $\CC\to\Set$ and
prove that one bijection is natural in $c$ (hence also its inverse)
    \item Extend $F\mapsto(\CC(c,\_)\To F)$ to a functor $[\CC\to\Set]\to\Set$ and
prove that one bijection is natural in $F$ (so also its inverse)
   \end{enumerate}
 \end{itemize}

 }

\frame
  {   
    \frametitle{Step 1, verification of the claimed inverse}\label{Yon:bijection}

 \begin{enumerate}[<+->]
\item Given $x\in F(c)$, define $\beta_d(g)=F(g)(x)$ for any object $d$ 
and $g\in\CC(c,d)$, then $\Phi_c(\beta)=\beta_c(id_c) =x$ and $\beta$ is natural:
\begin{tikzcd}[ampersand replacement=\&]
d \arrow[swap]{d}{f} \& \CC(c,d)  \arrow[swap]{d}{\_;f}\arrow{r}{\beta_d} \&\arrow{d}{F(f)} F(d)  \& 
g \arrow[swap, mapsto]{d}{\_;f}\arrow[mapsto]{r}{\beta_d} \&\arrow[mapsto]{d}{F(f)} F(g)(x) \\
e \& \CC(c,e) \arrow{r}{\beta_e} \& F(e)\& g;f \arrow[mapsto]{r}{\beta_e} \& {\color{red} ?}
\end{tikzcd}
Indeed ${\color{red} ?}=\beta_e(g;f)=F(g;f)(x) = F(f)(F(g)(x))$
\item Any natural $\alpha$ is completely determined by $\alpha_c(id_c)$, 
since $\alpha_d(f)=\alpha_d(id_c;f)=F(f)(\alpha_c(id_c))$:\\
\begin{tikzcd}[ampersand replacement=\&]
\redc \arrow[swap]{d}{f} \& \CC(c,\redc)  \arrow[swap]{d}{\_;f}\arrow{r}{\alpha_\redc} \&\arrow{d}{F(f)} F(\redc) \&
id_c \arrow[swap,mapsto]{d}{\_;f}\arrow[mapsto]{r}{\alpha_c} \&\arrow[mapsto]{d}{F(f)} \alpha_c(id_c)\\
d \& \CC(c,d) \arrow{r}{\alpha_d} \& F(d) \&
f \arrow[mapsto]{r}{\alpha_d} \& F(f)(\alpha_c(id_c))
\end{tikzcd}
 \end{enumerate}

 }

\frame
  {   
    \frametitle{Step 2, naturality in the object $c$}\label{Yon:natural_in_c}

 \begin{itemize}[<+->]
\item To extend $c\mapsto(\CC(c,\_)\To F)$ to a functor $\CC\to\Set$, define
for any $f:c\to d$ a map $(\CC(c,\_)\To F)\to(\CC(d,\_)\To F)$ by
$({\color{red}f(\alpha)})_t (h) = \alpha_t(f;h)$ for all $t$ and $h\in\CC(d,t)$ 
(NB {\color{red}notation}).
Functoriality: $id_c(\alpha) = \alpha$ and $(f;g)(\alpha) = g(f(\alpha))$ (why?)
\item Now we can verify naturality in $c$ in the following diagram:
\begin{tikzcd}[ampersand replacement=\&]
c \arrow[swap]{d}{f} \& \CC(c,\_)\To F  \arrow[swap]{d}{\color{red}f}\arrow{r}{\Phi_c} \&\arrow{d}{F(f)} F(c) \&
\alpha \arrow[swap,mapsto]{d}{\color{red}f}\arrow[mapsto]{r}{\Phi_c} \&\arrow[mapsto]{d}{F(f)} \alpha_c(id_c)\\
d \& \CC(d,\_)\To F \arrow{r}{\Phi_d} \& F(d) \&
{\color{red}f(\alpha)} \arrow[mapsto]{r}{\Phi_d} \& {\color{red}?}
\end{tikzcd}

Indeed ${\color{red} ?}=\Phi_d(f(\alpha)) = (f(\alpha))_d(id_d) = \alpha_d(f;id_d) = \alpha_d(f)$
and the diagram commutes as in 2.\ on the previous slide.
 \end{itemize}

 }

\frame
  {   
    \frametitle{Step 3, naturality in the functor $F$}\label{Yon:natural_in_F}

 \begin{itemize}[<+->]
\item Extend $F\mapsto(\CC(c,\_)\To F)$ to a functor $[\CC\to\Set]\to\Set$
by postcomposition with $\varphi:F\To G$: 
$\alpha \mapsto \alpha;\varphi$. %(NB {\color{red}notation})
%$(\CC(c,\_)\To F)\to(\CC(c,\_)\To G)$ by
Functoriality: 
$\alpha;id_F = \alpha$ and $\alpha;(\varphi;\psi) = (\alpha;\varphi);\psi$
\item Recall: $\Phi_c : (\CC(c,\_)\To F) \to F(c): \alpha\mapsto \alpha_c(id_c)$
\item Define: $\Psi_c : (\CC(c,\_)\To G) \to G(c): \beta\mapsto \beta_c(id_c)$
\item Now we can verify naturality in $F$ in the following diagram:
\begin{tikzcd}[ampersand replacement=\&]
F \arrow[swap]{d}{\varphi} \& \CC(c,\_)\To F  \arrow[swap]{d}{\_;\varphi}\arrow{r}{\Phi_c} \&\arrow{d}{\varphi_c} F(c) \&
\alpha \arrow[swap,mapsto]{d}{\_;\varphi}\arrow[mapsto]{r}{\Phi_c} \&\arrow[mapsto]{d}{\varphi_c} \alpha_c(id_c)\\
G \& \CC(c,\_)\To G \arrow{r}{\Psi_c} \& G(c) \&
{\alpha;\varphi} \arrow[mapsto]{r}{\Psi_d} \& {\color{red}?}
\end{tikzcd}

Indeed $\Psi_d(\alpha;\varphi) = (\alpha;\varphi)_c(id_c) = \varphi_c(\alpha_c(id_c))$
and the diagram commutes.
 \end{itemize}

 }

\frame
  {   
    \frametitle{A puzzling application of the Yoneda Lemma in Haskell}\label{Yon:Hask}

 \begin{itemize}[<+->]
\item Recall in the Haskell category $\HH$, with types as objects and 
functions as morphisms, the functor $[\_]:\HH\to\HH$ :
   \begin{itemize}[<+->]
\item mapping any type $T$ to the type $[T]$ of lists over $T$
\item mapping any $f: T\to T'$ to $map~f: [T] \to [T']$
   \end{itemize}
\item The Yoneda Lemma implies that any polymorphic function
$f$ of type $(Int\to t) \to [t]$, $t$ a type variable,
is completely given by a list of integers, which is $f(id_{Int})$!
\item Exercise: make sure you understand the details ($c,F,\CC(c,\_)$) % of the Yoneda Lemma
\item Discussion
 \end{itemize}

 }

\frame
  {   
    \frametitle{The Yoneda embedding}\label{Yon:Embed}

 \begin{itemize}[<+->]
\item The Yoneda embedding is a functor $Y: \CC^{op} \to [\CC\to\Set]$
with $Y(c)=\CC(c,\_)$
\item Still to define: $Y(f):\CC(c,\_)\To\CC(d,\_)$ for all $f\in\CC(d,c)$
\item Recall $\lambda {\color{red}x}.\lambda t. \lambda g\in\CC(c,t). F(g)(x) : {\color{red}F(c)} \to (\CC(c,\_)\To F)$,
which is a formal rendering of the inverse Yoneda bijection constructed in
point 1 of slide \ref{Yon:bijection}
\item Apply this to the functor $F= \CC(d,\_)$ and we get
$\lambda {\color{red}x}.\lambda t. \lambda g\in\CC(c,t). (x;g) : {\color{red}\CC(d,c)} \to (\CC(c,\_)\To \CC(d,\_))$
\item Precomposition with ${\color{red}x}$ is functorial, so we put $Y(f)= (f;\_)$
\item The Yoneda Lemma implies that $Y$ is full and faithful
\item Similarly: any small $\CC$ can be embedded in $[\CC^{op}\to\Set]$
 \end{itemize}

 }

\end{document}




\myurl{en.wikipedia.org/wiki/Ordered_pair}
 \begin{itemize}
    \item emulation halts because of the partiality of $\delta$;
    \item emulation halts because of "head left" with head at first cell;
    \item emulation halts because of reaching a halting state;
    \item the emulated TM goes on forever.
 \end{itemize}
