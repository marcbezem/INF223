\documentclass[handout]{beamer}
%\documentclass[slides]{beamer}
% Vary the color applet  (try out your own if you like)
%\colorlet{structure}{red!20!black}
%\beamertemplateshadingbackground{yellow!20}{white}
%\usepackage{beamerthemeshadow}
%\usepackage[utf8x]{inputenc} CONFLICT!
\usepackage{tikz}
%\usepackage[english,norsk,nynorsk]{babel}
\usepackage{tikz-cd}
\usetikzlibrary{trees}

\usepackage[all]{xy}
\usepackage{multicol}

%\setbeamertemplate{navigation symbols}{}++++++
%\setbeamertemplate{footline}[frame number]
\usetheme{Montpellier}


\input macros

\newcommand{\To}{\Rightarrow}
\newcommand{\Trt}{\stackrel{*}{\Rightarrow}}
\newcommand{\ToG}{\Rightarrow_G}
\newcommand{\redS}{{\color{red} S}}

\newcommand{\bfsf}[1]{{\boldsymbol{#1}}}
\newcommand{\Set}{\bfsf{Set}}
\newcommand{\Gra}{\bfsf{Graph}}
\newcommand{\CC}{\bfsf{C}}
\newcommand{\DD}{\bfsf{D}}
\newcommand{\EE}{\bfsf{E}}
\newcommand{\PP}{\bfsf{P}}
\newcommand{\HH}{\bfsf{H}}
\newcommand{\Nat}{\bfsf{Nat}}
\newcommand{\Incl}{\bfsf{Incl}}
\newcommand{\Rel}{\bfsf{Rel}}
\newcommand{\Mult}{\bfsf{Mult}}
\newcommand{\Mon}{\bfsf{Mon}}
\newcommand{\Cat}{\bfsf{Cat}}
\newcommand{\CAT}{\bfsf{CAT}}

\newcommand{\Kp}[1]{{\langle #1 \rangle}}
\newcommand{\Kc}{;\!;}
\newcommand{\bind}{{>}\!\!{>}\!{=}}
\newcommand{\ttt}[1]{\text{\tt #1}}

\title[INF223 presentations]{}

\begin{document}

\section{Monads}
\subsection{Introduction}
 
\frame
  {   
    \frametitle{You need a lemma}\label{Yon:Intro}

 \begin{itemize}[<+->]
\item Assume we have the following data:
\begin{itemize}
    \item a small category $\CC$
    \item a functor $F:\CC\to\Set$
    \item an object $c$ of $\CC$, so a functor $\CC(c,\_):\CC\to\Set$
 \end{itemize}
Then the set of natural transformations $\CC(c,\_)\To F$ is in bijective
correspondence with the set $F(c)$, and this in a `natural' way
(we will explain this and make this precise)
\item The above statement is called the Yoneda Lemma
\item Useful links are:
 \begin{itemize}
    \item \myurl{https://en.wikipedia.org/wiki/Yoneda\_lemma}
    \item \myurl{https://ncatlab.org/nlab/show/Yoneda+lemma}
 \end{itemize}
 \end{itemize}

 }

\frame
  {   
    \frametitle{Exploration of the simplest possible case}\label{Yon:Triv}

 \begin{itemize}[<+->]
\item Let $\CC$ be the trivial category, with only one object $1$ and
 its identity morphism $id_1$. Then $\CC(1,\_):\CC\to\Set$ maps
$1$ to $\CC(1,1) = \set{id_1}$ and $\CC(1,id_1) = id_{\set{id_1}}$
\item Let $F:\CC\to\Set$ be a functor, then $F(1)$ is a set, and
$F(id_1) = id_{F(1)}$
\item Any natural transformation $\alpha: \CC(c,\_)\To F$ consists of
only $\alpha_1: \set{id_1} \to F(1)$ and,
conversely, any such map is a natural transformation
since the only morphism in $\CC$ is $id_1$
\item Yoneda now says:  $(\set{id_1} \to F(1)) \simeq F(1)$ in $\Set$
\item Since $F$ was arbitrary we get  $(1_\Set \to X) \simeq X$ for every set $X$
\item Thus the Yoneda lemma is a (vast) generalization of this simple fact
 \end{itemize}

 }

\end{document}

\myurl{en.wikipedia.org/wiki/Ordered_pair}
 \begin{itemize}
    \item emulation halts because of the partiality of $\delta$;
    \item emulation halts because of "head left" with head at first cell;
    \item emulation halts because of reaching a halting state;
    \item the emulated TM goes on forever.
 \end{itemize}
