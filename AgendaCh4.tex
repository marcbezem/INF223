\documentclass[handout]{beamer}
%\documentclass[slides]{beamer}
% Vary the color applet  (try out your own if you like)
%\colorlet{structure}{red!20!black}
%\beamertemplateshadingbackground{yellow!20}{white}
%\usepackage{beamerthemeshadow}
%\usepackage[utf8x]{inputenc} CONFLICT!
%\usepackage[english,norsk,nynorsk]{babel}
\usepackage{tikz}
\usepackage{tikz-cd} 
\usetikzlibrary{trees}

\usepackage[all]{xy}
\usepackage{multicol}

%\setbeamertemplate{navigation symbols}{}++++++
%\setbeamertemplate{footline}[frame number]
\usetheme{Montpellier}
\setbeamertemplate{footline}[frame number]


\input macros

\newcommand{\To}{\Rightarrow}
\newcommand{\Trt}{\stackrel{*}{\Rightarrow}}
\newcommand{\ToG}{\Rightarrow_G}
\newcommand{\redS}{{\color{red} S}}

\newcommand{\bfsf}[1]{{\boldsymbol{#1}}}
\newcommand{\Set}{\bfsf{Set}}
\newcommand{\Gra}{\bfsf{Graph}}
\newcommand{\CC}{\bfsf{C}}
\newcommand{\DD}{\bfsf{D}}
\newcommand{\EE}{\bfsf{E}}
\newcommand{\PP}{\bfsf{P}}
\newcommand{\Nat}{\bfsf{Nat}}
\newcommand{\Incl}{\bfsf{Incl}}
\newcommand{\Rel}{\bfsf{Rel}}
\newcommand{\Mult}{\bfsf{Mult}}
\newcommand{\Mon}{\bfsf{Mon}}
\newcommand{\Cat}{\bfsf{Cat}}
\newcommand{\CAT}{\bfsf{CAT}}


\title[INF223 presentations]{}

\begin{document}

\section{Chapter 4}
\subsection{4.1 Equivalences and quotients}
 
\frame
  {   
    \frametitle{Equivalences}\label{Ch4:Eqvs}

 \begin{itemize}[<+->]
\item Important method of abstraction: make equality coarser, by making
some elements `equal' that were not equal before
\item Such a new `equality' called \emph{equivalence} should have some important 
properties that any equality relations has: reflexivity, symmetry and transitivity 
(\emph{refsymtra})
\item There are at least 3 ways to define an equivalence on a set $A$:
   \begin{enumerate}[<+->]
\item directly, by defining a refsymtra relation ${\simeq}\subseteq A\times A$
\item by partitioning $A = \cup_{i\in I} A_i $ in non-empty, disjoint, $I$-indexed $A_i$
\item by defining a (typing) map $f: A\to T$ to some set $T$
   \end{enumerate}
\item These three ways are logically equivalent (cf.\ MNF130): 
   \begin{enumerate}[<+->]
\item $[a]_\simeq = \set{x\in A\mid a\simeq x}$, 
$A/{\simeq}=\set{[a]_\simeq \mid a\in A}$, $[\_] : A \to A/{\simeq}$
\item $a\simeq a' $ iff $a,a' \in A_i$ is refsymtra; conversely, $I = A/{\simeq}$, $A_i = i$
\item $a\simeq a' $ iff $f(a)=f(a')$ is refsymtra; conv-ly,  $T = A/{\simeq}$, $f=[\_]$
   \end{enumerate}
\item If $f$ in 3 is surjective, the equivalence classes are the preimages
 \end{itemize}

 }

\frame
  {   
    \frametitle{Example: equality modulo}\label{Ch4:EqExa}

 \begin{itemize}[<+->]
\item Let $\zet$ be the set of integers and $k\in\zet$, $k>0$
\item We define $=_k$, equality modulo $k$ :
   \begin{enumerate}[<+->]
\item directly, $x =_k y$ iff $x-y$ is a multiple of $k$
\item partitioning $\zet = \cup_{i\in \set{0,...,k-1}} \zet_i $, $\zet_i = \set{kz+i \mid z\in\zet}$
\item by defining  $m_k: \zet\to \set{0,...,k-1}$, $m_k(z) = (z \mod k)$ 
   \end{enumerate}
\item Q: which operations/relations carry over from $\zet$ to $\zet/{=_k}$?
\item A: only those that are \emph{invariant} under ${=_k}$, i.e., are
independent of the choice of representative of a class
   \begin{itemize}[<+->]
\item for function $f$:  $f(z) = f(z')$ if $z =_k z'$ 
\item for relation $R$:  $R(z,u) \iff R(z',u')$    if $z =_k z'$ and $u =_k u'$
\item similarly for other arities
   \end{itemize}
\item Example: \emph{modulo arithmetic} (addition, multiplication)
\item {\color{red}NB}: $2^0 = 1 \neq_3 8 = 2^3$, whereas $0 =_3 3$ (no base 2 exp mod 3)
 \end{itemize}

 }

\frame
  {   
    \frametitle{More terminology on equivalences}\label{Ch4:Eqvs}

 \begin{itemize}[<+->]
\item The map $[\_] : A\to A/{\simeq}$ is called the \emph{quotient} map (`class of')
\item The equivalence relation defined by $f(a)=f(a')$  is called the \emph{kernel} of $f:A\to T$,
 denoted $ker(f)$
\item Map Theorem: if $f:A\to T$ is surjective and $g:A\to B$ is 
such that $ker(f)\subseteq ker(g)$,
then there is a unique $h: T\to B$ such that $f;h=g$ ($h$ blurs the  nuances 
that $f$ sees but not $g$)
\begin{tikzcd}[ampersand replacement=\&]
B \&  \arrow[swap]{l}{g} \arrow{r}{f} A  \& T \arrow[swap, bend left]{ll}{h}
\end{tikzcd} NB $T$ isomorph with $A/ker(f)$ in $\Set$
\item Proof uses the Axiom of Choice: for $t\in T$, 
pick $a$ with $f(a)=t$ and define $h(t)=g(a)$, independent of choice of $a$
\item Universal property of quotients: if $\simeq$ is an equivalence relation on $A$,
then for every $f: A\to B$  satifying $f(a)=f(a')$ for all $a\simeq a'$,  there exists a unique
$g: (A/{\simeq})\to B$ such that $g([a]) = f(a)$ for all $a\in A$, i.e., 
\begin{tikzcd}[ampersand replacement=\&]
B \&  \arrow[swap]{l}{f} \arrow{r}{[\_]} A  \& A/{\simeq} \arrow[swap, bend left]{ll}{g}
\end{tikzcd}
 \end{itemize}

 }

\frame
  {   
    \frametitle{More examples}\label{Ch4:EqMorExa}

 \begin{itemize}[<+->]
\item Classic: $(j,k)\simeq(m,n)$ iff $j+n = m+k$ gives $\zet=\nat/{\simeq}$
\item Logical equivalence
\item Isomorphy of objects of a category $\CC$ is an equivalence
\item Isomorphy of categories is an equivalence
\item Every category $\CC$ is isomorphic to a quotient of $\PP(gr(\CC))$,
the path category of its underlying graph (proof: take the quotient generated
by all path equalities $[id^\CC_A]=[]_A$ and $[f,g] = [f;g]$)
\item Equivalence of categories is an even coarser equivalence
 \end{itemize}

 }

\subsection{4.2 Special morphisms}
 
\frame
  {   
    \frametitle{Monomorphisms}\label{Ch4:Monos}

 \begin{itemize}[<+->]
\item A monomorphism is the categorical generalization of an injective map in $\Set$
\item Given a category $\CC$ and objects $A,B$, a morphism $f: A\to B$ is a
\emph{monomorphism} or \emph{mono} if for all objects $X$ and 
morphisms $g,h: X\to A$ we have $g=h$ if $g;{\color{red}f} = h;{\color{red}f}$
\item In diagrams: if
\begin{tikzcd}[ampersand replacement=\&]
%1 \arrow[yshift=0.5ex]{r}{y} \& 2, \arrow[yshift=-0.5ex]{l}{z}  \&
X \arrow[bend left]{r}{g}\arrow[bend right]{r}{h} \& 
A \arrow{r}{f}\& B
\end{tikzcd}
 commutes, then $g=h$
\item In $\Set$ this is equivalent with $f$ being injective:
 \begin{itemize}
    \item if $f$ injective, then indeed $f(g(x))=f(h(x))$ implies $g(x)=h(x)$ for all $x\in X$
    \item if $f$ mono and $f(a)=f(a')$, then 
for constant functions $c_a,c_{a'} : X\to A$,
 \begin{tikzcd}[ampersand replacement=\&]
X \arrow[bend left]{r}{c_a}\arrow[bend right]{r}{c_{a'}} \& 
A \arrow{r}{f}\& B
\end{tikzcd}
commutes, so $c_{a}=c_{a'}$ which implies $c_{a}(x)=c_{a'}(x)$ for all $x\in X$, so $a=a'$
{\color{red}provided we have an $x\in X $ (important proviso)}. Take, e.g., $X=\set{0}$.
\end{itemize}
 \end{itemize}

 }

\frame
  {   
    \frametitle{More on monos}\label{Ch4:Moronmonos}

 \begin{itemize}[<+->]
\item In preorder categories (such as $\Incl$) all morphisms are monos
\item In the category of sets with injections all morphisms are monos,
for a different reason as in the previous point
\item If $f:A\to B$ and $g:B\to C$ are monos, then so is $f;g$
\item If $f;g: A\to C$ is mono, then so is $f:A\to B$ (for any $g$)
\item Proofs: by `right' cancelling  and diagrammatic reasoning in
\[
\begin{tikzcd}[ampersand replacement=\&]
X \arrow[bend left]{r}{g}\arrow[bend right]{r}{h} \& 
A \arrow{r}{f}\& B \arrow{r}{g} \& C
\end{tikzcd}
\]
\item In $\Gra$, $\varphi: G\to H$ is a graph monomorphism iff both
maps $\varphi_V: G_V\to H_V$ and $\varphi_E: G_E\to H_E$ are monos in $\Set$
 \begin{itemize}
    \item if $\varphi_V,\varphi_E$ injective, and $\psi_x,\psi_y: X\to G$ in $\Gra$
with $\psi_x;\varphi = \psi_y;\varphi$, then we reason $(\__E,\__V)$-wise in $\Set$
    \item if $\varphi$ is mono, use the reasoning pattern  of the previous
slide with $X=(\set{0},\emptyset,\emptyset,\emptyset)$ for 
$\varphi_V$ and $X=gr({\cdot}{\to}{\cdot})$ for $\varphi_E$
\end{itemize}
\item Similar result for functors $F: \CC\to\DD$ in $\Cat$ (exercise)
 \end{itemize}

 }

\frame
  {   
    \frametitle{Monomorphisms in $\Rel$}\label{Ch4:MonosRel}

 \begin{itemize}[<+->]
\item How would a monomorhism in $\Rel$ look like?
\item 
If
\begin{tikzcd}[ampersand replacement=\&]
X \arrow[bend left]{r}{S}\arrow[bend right]{r}{T} \& 
A \arrow{r}{R}\& B
\end{tikzcd} %($S;R=T;R$)
commutes, then $S=T$
\item Recall  $\rho_R : A\to \powset{B}$, 
defined by $\rho_R(a) = \set{b\in B\mid R(a,b)}$,
in fact $\Rel$ and $\Mult$ are isomorphic in $\Cat$ (exercise)
\item Define $P_R : \powset{A}\to \powset{B}$ by $P_R(X) = \cup\set{\rho_R(x) \mid x\in X}$
%for all $X\subseteq A$
\item We have: $R$ is mono iff  $P_R : \powset{A}\to \powset{B}$ injective
 \begin{itemize}
    \item if $P_R$ injective, $S,T: X\to R$ in $\Rel$ with $S;R = T;R$, 
then for all $x\in X$, $P_R(\rho_S(x)) = P_R(\rho_T(x))$, so $\rho_S(x) = \rho_T(x)$ %, so $S=T$
    \item if $P_R$ is mono, $A_1,A_2\subseteq A$ with $P_R(A_1)=P_R(A_2)$,
find the right $X$ and $S_1,S_2: X\to A$ to prove $A_1=A_2$ (exercise)
\end{itemize}
\item 
 \end{itemize}

 }

\frame
  {   
    \frametitle{Epimorphisms}\label{Ch4:Epis}

 \begin{itemize}[<+->]
\item Given a category $\CC$ and objects $A,B$, a morphism $f: A\to B$ is an
\emph{epimorphism} or \emph{epi} if $f^{op}: B\to A$ is a mono in $\CC^{op}$
\item An epimorphism is the categorical generalization of an surjective map in $\Set$
({\color{red}`injective map' in $\Set^{op}$})
\item Rather than stopping here, we give the dual diagram:
for all objects $X$ and morphisms $g,h: A\to X$, we have $g=h$ if  
\begin{tikzcd}[ampersand replacement=\&]
%1 \arrow[yshift=0.5ex]{r}{y} \& 2, \arrow[yshift=-0.5ex]{l}{z}  \&
X  \& 
A \arrow[bend left]{l}{g}\arrow[bend right]{l}{h} \& B\arrow[swap]{l}{f}
\end{tikzcd}
 commutes, that is, if ${\color{red}f};g={\color{red}f};h$   %,`left cancellation')
\item In $\Set$ this is indeed equivalent with $f$ being surjective:
 \begin{itemize}
    \item if $f$ surjective, then indeed $g(f(b))=h(f(b)$ for all $b\in B$
      implies $g(a)=h(a)$ for all $a\in A$, since the image of $f$ is $A$
    \item if $f$ epi we take $X=\powset{\set{0}}$ and define $g,h: A\to X$ by $g(a)=\set{0}$
and {\color{red}$h(a)=\set{ x\in\set{0}\mid \exists b\in B (f(b)=a)}$ (tricky!).}
Clearly $f;g=f;h$, so $g=h$. Hence $f$ is surjective.
\end{itemize}
 \end{itemize}

 }

\frame
  {   
    \frametitle{Subobject classifiers}\label{Ch4:SubObj}

 \begin{itemize}[<+->]
\item Puzzled by $\set{ x\in\set{0}\mid \exists b\in B (f(b)=a)} \in \powset{\set{0}}$?
\item Actually, $\powset{\set{0}}$ is isomorphic to $\set{T,F}$ in $\Set$
and is a \emph{subset classifier}, a codomain for characteristic $\chi_A : B\to \set{T,F}$
for $A\subseteq B$, $\chi_A (b) = T$ if $b\in A$ and $\chi_A (b) = F$ otherwise
\item How would a \emph{subgraph classifier} look like? Like this $SGC$:
\begin{tikzcd}[ampersand replacement=\&]
out \arrow[loop left]{}{oo} \arrow[bend left]{r}{oi} \& 
in \arrow[bend left]{l}{io} \arrow[loop right,out = 15, looseness=30]{}{ii} \arrow[loop right,looseness=6]{}{out}
\end{tikzcd} (why $in$, $out$, $io$, etc ?)
\item Let $\varphi: G \to H$ in $\Gra$ be epi. Now define:
 \begin{itemize}
    \item $\psi : H\to SGC$ constant $in$ on nodes and $ii$ on arrows
    \item $\xi : H\to SGC$ $in/out$ on nodes in/out $\varphi$'s image,
     and $oo,io,oi$ on edges if enforced by  $\xi_V$, and $ii/out$ if the edge 
     is in/out $\varphi$'s image. Clearly $\varphi;\psi=\varphi;\xi$, so $\psi=\xi$.
\end{itemize}
It follows that $\varphi_V$ and  $\varphi_E$ are surjective in $\Set$; converse easy.
 \end{itemize}
% (beautifully)
 }

\frame
  {   
    \frametitle{Properties of monos and epis}\label{Ch4:MonandEp}

 \begin{itemize}[<+->]
\item Let $F: \CC\to\DD$ be a faithful functor (injective on morphisms),
then $F$ \emph{reflects} monos and epis in the following sense:
\item If $F(f)$ is a mono (epi) in $\DD$, then $f$ is a mono (epi) in $\CC$
\item Proof sketches: if $g;f=h;f$, then $F(g);F(f) = F(h); F(f)$, so $F(g)=F(h)$, so $g=h$;
$f;g=f;h$, then $F(f);F(g) =  F(f);F(h)$, so $F(g)=F(h)$, so $g=h$
\item Cor: isomorphism $F: \CC\to\DD$ preserve monos and epis
\item Exercise: show that $\varphi: G\to H$ is an epi in $\Gra$ iff
$\varphi_V$ and $\varphi_E$ are epis in $\Set$   
\item Not surprisingly, epis have the dual properties of monos
\item Example: if $f$ and $g$ are epis, then  $f^{op}$ and $g^{op}$ are monos,
so $(f;g)^{op}=g^{op} ; f^{op}$ is mono, so $f;g$ is epi
\item Exercise: prove that $g$ is epi if $f;g$ is epi
 \end{itemize}

 }

\frame
  {   
    \frametitle{Split monos and epis}\label{Ch4:SplitMonEp}

 \begin{itemize}[<+->]
\item For $f:A\to B$ and $g:B\to A$ in a given catergory $\CC$,
if $f;g=id_A$, then $f$ is mono and $g$ is epi
\item Proof: from previous results, since $id_A$ is both epi and mono
\item Definition: $f$ is a \emph{split} mono if $f;g=id_A$ for some $g$
\item Definition: $g$ is a \emph{split} epi if $f;g=id_A$ for some $f$
\item In $\Set$, every epi is split epi (equivalent to the Axiom of Choice)
\item Example: if $f$ and $g$ are epis, then  $f^{op}$ and $g^{op}$ are monos,
so $(f;g)^{op}=g^{op} ; f^{op}$ is mono, so $f;g$ is epi
\item Exercise: prove that $g$ is epi if $f;g$ is epi
 \end{itemize}

 }


\subsection{4.3 Special objects}
 
\frame
  {   
    \frametitle{Initial objects}\label{Ch4:InitialObj}

 \begin{itemize}[<+->]
\item object $i$ is initial in category $\CC$ if $\CC(i,x)$ a singleton for all objects $x$ of $\CC$
 \end{itemize}

 }

\frame
  {   
    \frametitle{Final objects}\label{Ch4:FinalObj}

 \begin{itemize}[<+->]
\item object $f$ is final in category $\CC$ if $\CC(x,f)$ a singleton for all objects $x$ of $\CC$
 \end{itemize}

 }


\frame
  {   
    \frametitle{Summary Chapter 4}\label{Ch4:Summary}

 \begin{enumerate}[<+->]
\item 
 \end{enumerate}


}

\frame
  {   
    \frametitle{Comments on script}\label{Ch4:comments}

 \begin{itemize}[<+->]
\item 
 \end{itemize}

 }

\end{document}

\myurl{en.wikipedia.org/wiki/Ordered_pair}
 \begin{itemize}
    \item emulation halts because of the partiality of $\delta$;
    \item emulation halts because of "head left" with head at first cell;
    \item emulation halts because of reaching a halting state;
    \item the emulated TM goes on forever.
 \end{itemize}
