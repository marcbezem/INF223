\documentclass[handout]{beamer}
%\documentclass[slides]{beamer}
% Vary the color applet  (try out your own if you like)
%\colorlet{structure}{red!20!black}
%\beamertemplateshadingbackground{yellow!20}{white}
%\usepackage{beamerthemeshadow}
%\usepackage[utf8x]{inputenc} CONFLICT!
%\usepackage[english,norsk,nynorsk]{babel}
\usepackage{tikz}
\usepackage{tikz-cd} 
\usetikzlibrary{trees}

\usepackage[all]{xy}
\usepackage{multicol}

%\setbeamertemplate{navigation symbols}{}++++++
%\setbeamertemplate{footline}[frame number]
\usetheme{Montpellier}
\setbeamertemplate{footline}[frame number]


\input macros

\newcommand{\To}{\Rightarrow}
\newcommand{\Trt}{\stackrel{*}{\Rightarrow}}
\newcommand{\ToG}{\Rightarrow_G}
\newcommand{\redS}{{\color{red} S}}

\newcommand{\bfsf}[1]{{\boldsymbol{#1}}}
\newcommand{\Set}{\bfsf{Set}}
\newcommand{\Gra}{\bfsf{Graph}}
\newcommand{\CC}{\bfsf{C}}
\newcommand{\DD}{\bfsf{D}}
\newcommand{\EE}{\bfsf{E}}
\newcommand{\PP}{\bfsf{P}}
\newcommand{\Nat}{\bfsf{Nat}}
\newcommand{\Incl}{\bfsf{Incl}}
\newcommand{\Rel}{\bfsf{Rel}}
\newcommand{\Mult}{\bfsf{Mult}}
\newcommand{\Mon}{\bfsf{Mon}}
\newcommand{\Cat}{\bfsf{Cat}}
\newcommand{\CAT}{\bfsf{CAT}}

\newcommand{\Kp}[1]{{\langle #1 \rangle}}


\title[INF223 presentations]{}

\begin{document}

\section{Chapter 4}
\subsection{4.1 Equivalences and quotients}
 
\frame
  {   
    \frametitle{Equivalences}\label{Ch4:Eqvs}

 \begin{itemize}[<+->]
\item Important method of abstraction: make equality coarser, by making
some elements `equal' that were not equal before
\item Such a new `equality' called \emph{equivalence} should have some important 
properties that any equality relations has: reflexivity, symmetry and transitivity 
(\emph{refsymtra})
\item There are at least 3 ways to define an equivalence on a set $A$:
   \begin{enumerate}[<+->]
\item directly, by defining a refsymtra relation ${\simeq}\subseteq A\times A$
\item by partitioning $A = \cup_{i\in I} A_i $ in non-empty, disjoint, $I$-indexed $A_i$
\item by defining a (typing) map $f: A\to T$ to some set $T$
   \end{enumerate}
\item These three ways are logically equivalent (cf.\ MNF130): 
   \begin{enumerate}[<+->]
\item $[a]_\simeq = \set{x\in A\mid a\simeq x}$, 
$A/{\simeq}=\set{[a]_\simeq \mid a\in A}$, $[\_] : A \to A/{\simeq}$
\item $a\simeq a' $ iff $a,a' \in A_i$ is refsymtra; conversely, $I = A/{\simeq}$, $A_i = i$
\item $a\simeq a' $ iff $f(a)=f(a')$ is refsymtra; conv-ly,  $T = A/{\simeq}$, $f=[\_]$
   \end{enumerate}
\item If $f$ in 3 is surjective, the equivalence classes are the preimages
 \end{itemize}

 }

\frame
  {   
    \frametitle{Example: equality modulo}\label{Ch4:EqExa}

 \begin{itemize}[<+->]
\item Let $\zet$ be the set of integers and $k\in\zet$, $k>0$
\item We define $=_k$, equality modulo $k$ :
   \begin{enumerate}[<+->]
\item directly, $x =_k y$ iff $x-y$ is a multiple of $k$
\item partitioning $\zet = \cup_{i\in \set{0,...,k-1}} \zet_i $, $\zet_i = \set{kz+i \mid z\in\zet}$
\item by defining  $m_k: \zet\to \set{0,...,k-1}$, $m_k(z) = (z \mod k)$ 
   \end{enumerate}
\item Q: which operations/relations carry over from $\zet$ to $\zet/{=_k}$?
\item A: only those that are \emph{invariant} under ${=_k}$, i.e., are
independent of the choice of representative of a class
   \begin{itemize}[<+->]
\item for function $f$:  $f(z) = f(z')$ if $z =_k z'$ 
\item for relation $R$:  $R(z,u) \iff R(z',u')$    if $z =_k z'$ and $u =_k u'$
\item similarly for other arities
   \end{itemize}
\item Example: \emph{modulo arithmetic} (addition, multiplication)
\item {\color{red}NB}: $2^0 = 1 \neq_3 8 = 2^3$, whereas $0 =_3 3$ (no base 2 exp mod 3)
 \end{itemize}

 }

\frame
  {   
    \frametitle{More terminology on equivalences}\label{Ch4:Eqvs}

 \begin{itemize}[<+->]
\item The map $[\_] : A\to A/{\simeq}$ is called the \emph{quotient} map (`class of')
\item The equivalence relation defined by $f(a)=f(a')$  is called the \emph{kernel} of $f:A\to T$,
 denoted $ker(f)$
\item Map Theorem: if $f:A\to T$ is surjective and $g:A\to B$ is 
such that $ker(f)\subseteq ker(g)$,
then there is a unique $h: T\to B$ such that $f;h=g$ ($h$ blurs the  nuances 
that $f$ sees but not $g$)
\begin{tikzcd}[ampersand replacement=\&]
B \&  \arrow[swap]{l}{g} \arrow{r}{f} A  \& T \arrow[swap, bend left]{ll}{h}
\end{tikzcd} NB $T$ isomorphic to $A/ker(f)$ in $\Set$
\item Proof uses the Axiom of Choice: for $t\in T$, 
pick $a$ with $f(a)=t$ and define $h(t)=g(a)$, independent of choice of $a$
\item Universal property of quotients: if $\simeq$ is an equivalence relation on $A$,
then for every $f: A\to B$  satifying $f(a)=f(a')$ for all $a\simeq a'$,  there exists a unique
$g: (A/{\simeq})\to B$ such that $g([a]) = f(a)$ for all $a\in A$, i.e., 
\begin{tikzcd}[ampersand replacement=\&]
B \&  \arrow[swap]{l}{f} \arrow{r}{[\_]} A  \& A/{\simeq} \arrow[swap, bend left]{ll}{g}
\end{tikzcd}
 \end{itemize}

 }

\frame
  {   
    \frametitle{More examples}\label{Ch4:EqMorExa}

 \begin{itemize}[<+->]
\item Classic: $(j,k)\simeq(m,n)$ iff $j+n = m+k$ gives $\zet=(\nat\times\nat)/{\simeq}$
\item Logical equivalence
\item Isomorphy of objects of a category $\CC$ is an equivalence
\item Isomorphy of categories is an equivalence
\item Every category $\CC$ is isomorphic to a quotient of $\PP(gr(\CC))$,
the path category of its underlying graph (proof: take the quotient generated
by all path equalities $[id^\CC_A]=[]_A$ and $[f,g] = [f;g]$)
\item Equivalence of categories is an even coarser equivalence
 \end{itemize}

 }

\subsection{4.2 Special morphisms}
 
\frame
  {   
    \frametitle{Monomorphisms}\label{Ch4:Monos}

 \begin{itemize}[<+->]
\item A monomorphism is the categorical generalization of an injective map in $\Set$
\item Given a category $\CC$ and objects $A,B$, a morphism $f: A\to B$ is a
\emph{monomorphism} or \emph{mono} if for all objects $X$ and 
morphisms $g,h: X\to A$ we have $g=h$ if $g;{\color{red}f} = h;{\color{red}f}$
\item In diagrams: if
\begin{tikzcd}[ampersand replacement=\&]
%1 \arrow[yshift=0.5ex]{r}{y} \& 2, \arrow[yshift=-0.5ex]{l}{z}  \&
X \arrow[bend left]{r}{g}\arrow[bend right]{r}{h} \& 
A \arrow{r}{f}\& B
\end{tikzcd}
 commutes, then $g=h$
\item In $\Set$ this is equivalent with $f$ being injective:
 \begin{itemize}
    \item if $f$ injective, then indeed $f(g(x))=f(h(x))$ implies $g(x)=h(x)$ for all $x\in X$
    \item if $f$ mono and $f(a)=f(a')$, then 
for constant functions $c_a,c_{a'} : X\to A$,
 \begin{tikzcd}[ampersand replacement=\&]
X \arrow[bend left]{r}{c_a}\arrow[bend right]{r}{c_{a'}} \& 
A \arrow{r}{f}\& B
\end{tikzcd}
commutes, so $c_{a}=c_{a'}$ which implies $c_{a}(x)=c_{a'}(x)$ for all $x\in X$, so $a=a'$
{\color{red}provided we have an $x\in X $ (important proviso)}. Take, e.g., $X=\set{0}$.
\end{itemize}
 \end{itemize}

 }

\frame
  {   
    \frametitle{More on monos}\label{Ch4:Moronmonos}

 \begin{itemize}[<+->]
\item In preorder categories (such as $\Incl$) all morphisms are monos
\item In the category of sets with injections all morphisms are monos,
for a different reason as in the previous point
\item If $f:A\to B$ and $g:B\to C$ are monos, then so is $f;g$
\item If $f;g: A\to C$ is mono, then so is $f:A\to B$ (for any $g$)
\item Proofs: by `right' cancelling  and diagrammatic reasoning in
\[
\begin{tikzcd}[ampersand replacement=\&]
X \arrow[bend left]{r}{h}\arrow[bend right]{r}{h'} \& 
A \arrow{r}{f}\& B \arrow{r}{g} \& C
\end{tikzcd}
\]
\item In $\Gra$, $\varphi: G\to H$ is a graph monomorphism iff both
maps $\varphi_V: G_V\to H_V$ and $\varphi_E: G_E\to H_E$ are monos in $\Set$
 \begin{itemize}
    \item if $\varphi_V,\varphi_E$ injective, and $\psi_x,\psi_y: X\to G$ in $\Gra$
with $\psi_x;\varphi = \psi_y;\varphi$, then we reason $(\__E,\__V)$-wise in $\Set$
    \item if $\varphi$ is mono, use the reasoning pattern  of the previous
slide with $X=(\set{0},\emptyset,...)$ for 
$\varphi_V$ and $X=gr({\cdot}{\to}{\cdot})$ for $\varphi_E$
\end{itemize}
\item Similar result for functors $F: \CC\to\DD$ in $\Cat$ (exercise)
 \end{itemize}

 }

\frame
  {   
    \frametitle{Monomorphisms in $\Rel$}\label{Ch4:MonosRel}

 \begin{itemize}[<+->]
\item How would a monomorhism in $\Rel$ look like?
\item Recall $\Rel$ is the category of sets with relations as morphisms
\item $R$ mono: if
\begin{tikzcd}[ampersand replacement=\&]
X \arrow[bend left]{r}{S}\arrow[bend right]{r}{T} \& 
A \arrow{r}{R}\& B
\end{tikzcd} %($S;R=T;R$)
commutes, then $S=T$
\item Recall  $\rho_R : A\to \powset{B}$, 
defined by $\rho_R(a) = \set{b\in B\mid R(a,b)}$,
in fact $\Rel$ and $\Mult$ are isomorphic in $\Cat$ (exercise)
\item Define $P_R : \powset{A}\to \powset{B}$ by $P_R(Z) = \cup\set{\rho_R(z) \mid z\in Z}$
%for all $X\subseteq A$
\item We have: $R$ is mono iff  $P_R : \powset{A}\to \powset{B}$ injective
 \begin{itemize}
    \item if $P_R$ injective, $S,T: X\to R$ in $\Rel$ with $S;R = T;R$, 
then for all $x\in X$, $P_R(\rho_S(x)) = P_R(\rho_T(x))$, so $\rho_S(x) = \rho_T(x)$, so $S=T$
    \item if $P_R$ is mono, $A_1,A_2\subseteq A$ with $P_R(A_1)=P_R(A_2)$,
find the right $X$ and $S_1,S_2: X\to A$ to prove $A_1=A_2$ (exercise)
\end{itemize}

 \end{itemize}

 }

\frame
  {   
    \frametitle{Epimorphisms}\label{Ch4:Epis}

 \begin{itemize}[<+->]
\item Given a category $\CC$ and objects $A,B$, a morphism $f: A\to B$ is an
\emph{epimorphism} or \emph{epi} if $f^{op}: B\to A$ is a mono in $\CC^{op}$
\item An epimorphism is the categorical generalization of an surjective map in $\Set$
({\color{red}`injective pam' in $\Set^{op}$})
\item Rather than stopping here, we give the dual diagram:
for all objects $X$ and morphisms $g,h: A\to X$, we have $g=h$ if  
\begin{tikzcd}[ampersand replacement=\&]
%1 \arrow[yshift=0.5ex]{r}{y} \& 2, \arrow[yshift=-0.5ex]{l}{z}  \&
X  \& 
A \arrow[bend left]{l}{g}\arrow[bend right]{l}{h} \& B\arrow[swap]{l}{f}
\end{tikzcd}
 commutes, that is, if ${\color{red}f};g={\color{red}f};h$   %,`left cancellation')
\item In $\Set$ this is indeed equivalent with $f$ being surjective:
 \begin{itemize}
    \item if $f$ surjective, then indeed $g(f(b))=h(f(b)$ for all $b\in B$
      implies $g(a)=h(a)$ for all $a\in A$, since the image of $f$ is $A$
    \item if $f$ epi we take $X=\powset{\set{0}}$ and define $g,h: A\to X$ by $g(a)=\set{0}$
and {\color{red}$h(a)=\set{ x\in\set{0}\mid \exists b\in B (f(b)=a)}$ (tricky!).}
Clearly $f;g=f;h$, so $g=h$. Hence $f$ is surjective.
\end{itemize}
 \end{itemize}

 }

\frame
  {   
    \frametitle{Subobject classifiers}\label{Ch4:SubObj}

 \begin{itemize}[<+->]
\item Puzzled by $\set{ x\in\set{0}\mid \exists b\in B (f(b)=a)} \in \powset{\set{0}}$?
\item Actually, $\powset{\set{0}}$ is isomorphic to $\set{T,F}$ in $\Set$
and is a \emph{subset classifier}, a codomain for characteristic $\chi_A : B\to \set{T,F}$
for $A\subseteq B$, $\chi_A (b) = T$ if $b\in A$ and $\chi_A (b) = F$ otherwise
\item How would a \emph{subgraph classifier} look like? Like this $SGC$:
\begin{tikzcd}[ampersand replacement=\&]
out \arrow[loop left]{}{oo} \arrow[bend left]{r}{oi} \& 
in \arrow[bend left]{l}{io} \arrow[loop right,out = 15, looseness=30]{}{ii} \arrow[loop right,looseness=6]{}{out}
\end{tikzcd} (why $in$, $out$, $io$, etc ?)
\item Let $\varphi: G \to H$ in $\Gra$ be epi. Now define:
 \begin{itemize}
    \item $\psi : H\to SGC$ constant $in$ on nodes and $ii$ on arrows
    \item $\xi : H\to SGC$ $in/out$ on nodes in/out $\varphi$'s image,
     and $oo,io,oi$ on edges if enforced by  $\xi_V$, and $ii/out$ if the edge 
     is in/out $\varphi$'s image. Clearly $\varphi;\psi=\varphi;\xi$, so $\psi=\xi$.
\end{itemize}
It follows that $\varphi_V$ and  $\varphi_E$ are surjective in $\Set$; converse easy.
 \end{itemize}
% (beautifully)
 }

\frame
  {   
    \frametitle{Properties of monos and epis}\label{Ch4:MonandEp}

 \begin{itemize}[<+->]
\item Let $F: \CC\to\DD$ be a faithful functor (injective on morphisms),
then $F$ \emph{reflects} monos and epis in the following sense:
\item If $F(f)$ is a mono (epi) in $\DD$, then $f$ is a mono (epi) in $\CC$
\item Proof sketches: if $g;f=h;f$, then $F(g);F(f) = F(h); F(f)$, so $F(g)=F(h)$, so $g=h$;
$f;g=f;h$, then $F(f);F(g) =  F(f);F(h)$, so $F(g)=F(h)$, so $g=h$
\item COR isomorphism $F: \CC\to\DD$ preserve monos and epis
%\item Exercise: show that $\varphi: G\to H$ is an epi in $\Gra$ iff
%$\varphi_V$ and $\varphi_E$ are epis in $\Set$   
\item Not surprisingly, epis have the dual properties of monos
\item Example: if $f$ and $g$ are epis, then  $f^{op}$ and $g^{op}$ are monos,
so $(f;g)^{op}=g^{op} ; f^{op}$ is mono, so $f;g$ is epi
\item Exercise: prove that $g$ is epi if $f;g$ is epi
 \end{itemize}

 }

\frame
  {   
    \frametitle{Split monos and epis}\label{Ch4:SplitMonEp}

 \begin{itemize}[<+->]
\item For $f:A\to B$ and $g:B\to A$ in a given category $\CC$,
if $f;g=id_A$, then $f$ is mono and $g$ is epi
\item Proof: from previous results, since $id_A$ is both epi and mono
\item Definition: $f$ is a \emph{split} mono if $f;g=id_A$ for some $g$
\item Definition: $g$ is a \emph{split} epi if $f;g=id_A$ for some $f$
\item In $\Set$, every epi is split epi (equivalent to the Axiom of Choice)
\item If $f:A\to B$ is an iso with inverse $g:B\to A$, then both $f$ and $g$ are
both split mono and split epi
\item Proof: use that $f;g = id_A$  and $g;f = id_B$
 \end{itemize}

 }


\subsection{4.3 Special objects}
 
\frame
  {   
    \frametitle{Initial objects}\label{Ch4:InitialObj}

 \begin{itemize}[<+->]
\item object $I$ of category $\CC$ is \emph{initial}  
if $\CC(I,X)$ a singleton for all objects $X$ of $\CC$, 
notation $\CC({\color{red}I},X)=\set{{\color{red}0_X}}$ with $0_X: I\to X$
\item Examples of initial objects: 
 \begin{itemize}
\item $\emptyset$ in $\Set$, 
with $\Set(\emptyset,A)=\set{in_{\emptyset,A}}$ %for any $A$
\item $\emptyset$ in $\Rel$ and $\Mult$ as well, 
with $\Rel(\emptyset,A) = \set{\emptyset} $ (why?) 
and $ \Mult(\emptyset,A) = \set{in_{\emptyset,\powset A}}$
\item$(\emptyset,\emptyset,id_{\emptyset},id_{\emptyset})$
in $\Gra$, the unit monoid in $\Mon$, the empty category in $\Cat$, etc.
 \end{itemize} 
\item LEM if $0$ is initial in $\CC$, then $0_0 = id_0$, and $0_X;f = 0_Y$ for 
all $f: X\to Y$, $0_X;f = id_0$ if $Y=0$,
and $I\simeq 0$ iff $I$ is
initial in $\CC$, for all $I$
\item LEM initiality of objects is preserved by isomorphisms and
equivalences of categories
\item Exercise: prove the lemmas above
\end{itemize}

 }

\frame
  {   
    \frametitle{Final objects, the dual notion of initial object}\label{Ch4:FinalObj}

 \begin{itemize}[<+->]
\item object $F$ of category $\CC$ is \emph{final} 
if $\CC(X,F)$ a singleton for all objects $X$ of $\CC$, 
notation $\CC(X,{\color{red}F})=\set{{\color{red}1_X}}$, with $1_X : X\to F$
\item Examples of final objects: 
 \begin{itemize}
\item any singleton set in $\Set$, 
with, e.g., $\Set(A,\set{0})=\set{a\mapsto 0}$ %for any $A$
\item surprise: $\emptyset$ in $\Rel$ and $\Mult$ (why?), 
with $\Rel(A,\emptyset) = \set{\emptyset} $ (why?) 
and $ \Mult(A,\emptyset) = \set{a\mapsto \emptyset}$
\item$(\set{0},\set{0},id_{\set{0}},id_{\set{0}})$
in $\Gra$, the unit monoid in $\Mon$ (surprise, why?), 
which is also final in $\Cat$ (why?), etc.
 \end{itemize} 
\item LEM if $1$ is final in $\CC$, then $1_1 = id_1$ 
and $f;1_Y = 1_X$ for all $f: X\to Y$, $f;1_Y = id_1$ if $X=1$,
and $F\simeq 1$ iff $F$ is final in $\CC$
\item LEM finality of objects is preserved by isomorphisms and
equivalences of categories
\item Exercise: prove the lemmas above
 \end{itemize}

 }

\subsection{4.4 Universal constructions}
 
\frame
  {   
    \frametitle{Universal properties}\label{Ch4:UnivProp}

 \begin{itemize}[<+->]
\item The definition "object $I$ of category $\CC$ is \emph{initial}  
if for all objects $X$ of $\CC$ there is a unique morphism $I\to X$" characterizes  
initial objects up to unique isomorphism by what is called a \emph{universal property}
\item One can say it is a universal construction of arity  0 (no input)
\item Now: two universal constructions of arity 2 (two inputs)
\begin{itemize}
    \item sum (or coproduct), generalising disjoint sum of sets
    \item product, generalising cartesian product of sets
 \end{itemize}
They are each other's dual, and we show the diagrams next %(next slide)
\item Seen already (in $Set$):
\begin{itemize}
    \item a universal property of lists
    \item a universal property of quotients
 \end{itemize}
\item NB all are properties that a category may or may not have!
\end{itemize}

 }

\frame
  {   
    \frametitle{Sum and product}\label{Ch4:SumProd}

 \begin{itemize}[<+->]
\item Diagrams for sum (coproduct) and product of given objects $A$ and $B$
in an arbitrary category $\CC$ :
\[
\begin{tikzcd}[ampersand replacement=\&]
A \arrow{r}{\kappa_1} \arrow[swap]{dr}{f_1}\&
C\arrow{d}[description]{\exists!k} \& 
B \arrow[swap]{l}{\kappa_2}\arrow{dl}{f_2} \&
A \&
C\arrow[swap]{l}{\pi_1} \arrow{r}{\pi_2}\&
B\\
\& X\&\& \&
X\arrow{ul}{f_1}\arrow{u}[description]{\exists!k} \arrow[swap]{ur}{f_2}
\end{tikzcd}
\]
\item Sum in $\CC$: a co-span $A\stackrel{\kappa_1}{\to} C \stackrel{\kappa_2}{\leftarrow}B$ 
is called a \emph{sum} of $A$ and $B$ 
if for any co-span 
$A\stackrel{f_1}{\to} X \stackrel{f_2}{\leftarrow}B$ 
there is a unique morphism $k: C\to X$
such that $f_1 = \kappa_1;k$ and $f_2 = \kappa_2;k$

\item Product in $\CC$: a span $A\stackrel{\pi_1}{\leftarrow} C \stackrel{\pi_2}{\to}B$ 
is called a \emph{product} of $A$ and $B$ if for any span 
$A\stackrel{f_1}{\leftarrow} X \stackrel{f_2}{\to}B$ 
there is a unique morphism $k: X\to C$
such that $f_1 = k;\pi_1$ and $f_2 = k;\pi_2$

 \end{itemize}

 }

\frame
  {   
    \frametitle{Examples of sums}\label{Ch4:SumExa}

% \begin{itemize}[<+->]
%\item 
In the category $\Set$, the co-span 
$A\stackrel{\kappa_1}{\to} A\uplus B \stackrel{\kappa_2}{\leftarrow}B$,
with $\kappa_1(a) = (a,1)$ and $\kappa_2(b) = (b,2)$ has the universal property
\begin{tikzcd}[ampersand replacement=\&]
A \arrow{r}{\kappa_1} \arrow[swap]{dr}{f_1}\&
A\uplus B\arrow{d}[description]{\exists!k} \& 
B \arrow[swap]{l}{\kappa_2}\arrow{dl}{f_2} \\
\& X\&
\end{tikzcd}
by $k(p) =\left\{\begin{array}{ll} f_1(a) & \text{if $p=(a,1)$}\\f_2(b) &  \text{if $p=(b,2)$}\end{array}\right.$\\\vspace*{0.3cm}
%\end{itemize}
In the category $\Rel$, the co-span 
$A\stackrel{K_1}{\to} A\uplus B \stackrel{K_2}{\leftarrow}B$,
with (relations)
$K_1 = \set{(a,(a,1))\mid a\in A}$ and $K_2 = ...$  has the universal property
\begin{tikzcd}[ampersand replacement=\&]
A \arrow{r}{K_1} \arrow[swap]{dr}{R_1}\&
A\uplus B\arrow{d}[description]{\exists!K} \& 
B \arrow[swap]{l}{K_2}\arrow{dl}{R_2} \\
\& X\&
\end{tikzcd}
$K=\set{((c,i),x ) \mid i\in\set{1,2}, R_i(c,x)}$ %iff  $\left\{\begin{array}{ll} R_x(a,q) & \text{if $p=(a,1)$}\\R_y(b,q) &  \text{if $p=(b,2)$}\end{array}\right.$
 }

\frame
  {   
    \frametitle{More examples of sums}\label{Ch4:MoreSums}

In the category $\Incl$, the preorder category of sets with inclusions,
the co-span 
$A\stackrel{\kappa_1}{\to} A\cup B \stackrel{\kappa_2}{\leftarrow}B$,
with $\kappa_1= in_{A,A\cup B}$ and $\kappa_2= in_{B,A\cup B}$ has the universal property
\begin{tikzcd}[ampersand replacement=\&]
A \arrow{r}{\kappa_1} \arrow[swap]{dr}{f_1}\&
A\cup B\arrow{d}[description]{\exists!k} \& 
B \arrow[swap]{l}{\kappa_2}\arrow{dl}{f_2} \\
\& X\&
\end{tikzcd}
by $k =  in_{A\cup B,X}$, since we have $A\cup B\subseteq X$ when
$A\subseteq X$ and $B\subseteq X$.
\\\vspace*{0.2cm}
This example shows how important the morphisms are: $\Incl$
and $\Set$ have different sums since the morphisms are different, 
even though they have the same objects ({\color{red}why not sum $A\uplus B$ in $\Incl$?})
\\\vspace*{0.2cm}
Not every preorder category  has sums, since sums are 
\emph{least upper bounds}, and they need not exist.
For example, $\bfsf{2\!\!2}$ has sums $1+1 = 1$,
$2+2=2$, but no sum $1+2$ (no lub of $1$ and $2$, {\color{red}why?})

 }

\frame
  {   
    \frametitle{Basic properties of sums}\label{Ch4:SumProp}

 \begin{itemize}[<+->]
\item Let $A$ and $B$ be objects of a category $\CC$. 
If their sum exists, we denote it by $A+B$ (leaving the $\kappa_i$ often implicit). 
The unique morphisms $k:  A+B \to X$ in the universal property are
called \emph{mediating} morphisms and denoted by $[\_,\_]$
%as in the diagrams below.
\item Postcomposition with $g: X\to Y$ gives this:
\[
\begin{tikzcd}[ampersand replacement=\&]
A \arrow{r}{\kappa_1}\arrow{dr}{f_1} \arrow[swap]{ddr}{f_1;g}\&
A+B\arrow{d}[description]{[f_1,f_2]}\& 
B \arrow[swap]{l}{\kappa_2}\arrow[swap]{dl}{f_2} \arrow{ddl}{f_2;g}\\
\& X \arrow{d}[description]{g}\\
\& Y
\end{tikzcd}
\]
\item By uniqueness we get $[f_1;g,f_2;g] = [f_1,f_2] ; g$, {\color{red}why?}
\item Exercise: show that $[\kappa_1,\kappa_2] = id_{A+B}$
 \end{itemize}

 }


\frame
  {   
    \frametitle{Uniqueness of sums up to unique isomorphism}\label{Ch4:SumUnique}

 %\begin{itemize}[<+->]
%\item 
If also the co-span $A\stackrel{f_1}{\to} X \stackrel{f_2}{\leftarrow}B$
is a sum we have:
\[
\begin{tikzcd}[ampersand replacement=\&]
A \arrow{r}{\kappa_1} \arrow[swap]{dr}{f_1}\&
C\arrow{d}[description]{\exists!k} \& 
B \arrow[swap]{l}{\kappa_2}\arrow{dl}{f_2} \&
A \arrow{r}{\kappa_1} \arrow[swap]{dr}{f_1}\&
C \& 
B \arrow[swap]{l}{\kappa_2}\arrow{dl}{f_2} \\
\& X\&\& \&
X\arrow{u}[description]{\exists!j}
\end{tikzcd}
\]
and hence $j;k = id_X$ and $k;j = id_C$ by uniqueness in: % the outer diagrams:
\[
\begin{tikzcd}[ampersand replacement=\&]
\& X\arrow{d}[description]{j}\&\& \&
C\arrow{d}[description]{k}\\
A \arrow{r}{\kappa_1} \arrow[swap]{dr}{f_1} \arrow{ur}{f_1}\&
C\arrow{d}[description]{k} \& 
B \arrow[swap]{l}{\kappa_2}\arrow{dl}{f_2} \arrow[swap]{ul}{f_2}\&
A \arrow{ru}{\kappa_1} \arrow{r}{f_1} \arrow[swap]{dr}{\kappa_1}\&
X \arrow{d}[description]{j}\& 
B \arrow[swap]{l}{f_2}\arrow{dl}{\kappa_2} \arrow[swap]{ul}{\kappa_2}\\
\& X\&\& \&
C%\arrow{u}[description]{\exists!j}
\end{tikzcd}
\]
so $X$ and $C$ are isomorphic in $\CC$ (if fact the co-spans are!)
 %\end{itemize}

 }

\frame
  {   
    \frametitle{Advanced properties of sums}\label{Ch4:SumPropAdv}

Sums in the category $\Gra$ are based on sums in $\Set$.
If $G=(G_V,G_E,sc^G,tg^G)$ and $H=(H_V,H_E,sc^H,tg^H)$,
then $G+H  = (G_V \uplus H_V,G_E \uplus H_E,sc^G{+}sc^H,tg^G{+}tg^H)$,
where (fancy notation) $f_1{+}f_2 : A_1\uplus A_2 \to B_1\uplus B_2$ is defined case-wise.\\\vspace*{0.3cm}

If category $\CC$ has sums, then so has the interpretation category
$[G\to\CC]$, for any graph $G$\\\vspace*{0.3cm}

If category $\CC$ has sums, and $T$ is an object of $\CC$, then the slice category
$\CC/T$ has sums: $(A+B, [f_1,f_2])$ is an object of $\CC/T$,
the $\kappa_i$ are morphisms, and the universal property holds ({\color{red}why?})
\\\vspace*{0.3cm}
Exercise: elaborate the sum $(A,f_1)+(B,f_2)$ in $\CC/T$,
in particular the universal property given $(X,g)$ and $h_1: (A,f_1)\to(X,g)$,
$h_2: (B,f_2)\to(X,g)$


 }

\frame
  {   
    \frametitle{Products}\label{Ch4:Prod}

 \begin{itemize}[<+->]
\item The product (left) of given objects $A$ and $B$
in an arbitrary category $\CC$ is the dual of the sum (right):
\[
\begin{tikzcd}[ampersand replacement=\&]
A \&
C\arrow[swap]{l}{\pi_1} \arrow{r}{\pi_2}\&
B  \&
A \arrow{r}{\kappa_1} \arrow[swap]{dr}{f_1}\&
C\arrow{d}[description]{\exists!k} \& 
B \arrow[swap]{l}{\kappa_2}\arrow{dl}{f_2}
\\
\& X\arrow{ul}{f_1}\arrow{u}[description]{\exists!k} \arrow[swap]{ur}{f_2}\&\&
\& X 
\end{tikzcd}
\]
\item All properties of the sum can be dualized to hold for the product
as well, in particular uniqueness 
\item If a product exists, 
we denote it by $A\times B$ (leaving the $\pi_i$ often implicit).
The unique morphisms $k:  X\to A\times B$ in the universal property are again
called \emph{mediating} morphisms and denoted by $\langle\_,\_\rangle$;
by uniqueness $\langle\pi_1,\pi_2\rangle= id_{A\times B}$

\item If $A\times B$ exists and $g: Y\to X$,
then $g;\langle f_1,f_2\rangle = \langle g;f_1,g;f_2\rangle$

 \end{itemize}

 }

\frame
  {   
    \frametitle{Examples of products}\label{Ch4:ProdExa}

% \begin{itemize}[<+->]
%\item 
In the category $\Set$, the span 
$A \stackrel{\pi_1}{\leftarrow} A\times B \stackrel{\pi_2}{\to}  B$,
with usual $\times$, $\pi_1(a,b) = a$ and $\pi_2(a,b) = b$ has the universal property
\begin{tikzcd}[ampersand replacement=\&]
A \&
A\times B\arrow[swap]{l}{\pi_1} \arrow{r}{\pi_2}\&
B  \&
\\
\& X\arrow{ul}{f_1}\arrow{u}[description]{\exists!k} \arrow[swap]{ur}{f_2}
\end{tikzcd}
by $k(x) =(f_1(x),f_2(x))$\\\vspace*{0.2cm}
%\end{itemize}

In the category $\Incl$, the preorder category of sets with inclusions,
the span 
$A \stackrel{\pi_1}{\leftarrow} A\cap B \stackrel{\pi_2}{\to}  B$,
with $\pi_1= in_{A\cap B,A}$, $\pi_2= in_{A\cap B,B}$ has the universal property
\begin{tikzcd}[ampersand replacement=\&]
A \&
A\cap B\arrow[swap]{l}{\pi_1} \arrow{r}{\pi_2}\&
B
\\
\& X\arrow{ul}{f_1}\arrow{u}[description]{\exists!k} \arrow[swap]{ur}{f_2}
\end{tikzcd}
by $k =  in_{X,A\cap B}$, since we have $X \subseteq A\cap B$ when
$X\subseteq A$ and $X\subseteq B$.

Not every preorder category  has products, since products are 
\emph{greatest lower bounds}, and they need not exist!
 }

\frame
  {   
    \frametitle{Surprise: products in $\Rel$}\label{Ch4:RelProd}

% \begin{itemize}[<+->]
%\item
In the category $\Rel$, the following is {\color{red}plainly wrong: $A\times B$ with
%the span $A \stackrel{\pi_1}{\leftarrow} A\times B \stackrel{\pi_2}{\to}  B$, with (relations)
$\pi_1 = \set{((a,b),a)\mid a\in A, b\in B}$, $\pi_2 =  \set{((a,b),b)\mid a\in A, b\in B}$  has the universal property
\begin{tikzcd}[ampersand replacement=\&]
A \&
A\times B\arrow[swap]{l}{\pi_1} \arrow{r}{\pi_2}\&
B\\
\& X\arrow{ul}{R_1}\arrow{u}[description]{\exists!K} \arrow[swap]{ur}{R_2}
\end{tikzcd}
by
$K=\set{(x,(a,b)) \mid x\in X, a\in A, b\in B, R_1(x,a), R_2(x,b)}$}
\\\vspace*{0.2cm}
Counterexample: $R_1$ empty and $R_2$ not empty! ({\color{red}why?})
\\\vspace*{0.2cm}
We could have known better: $\Rel\simeq\Rel^{op}$ with an iso $c$ that 
is the identity on objects and converses relations. In $\Rel$ we have sum
$A\stackrel{K_1}{\to} A\uplus B \stackrel{K_2}{\leftarrow}B$
as on slide \ref{Ch4:SumExa}. Isos preserve sums, so
we have in $\Rel^{op}$ a sum
$A\stackrel{K^c_1}{\to} A\uplus B \stackrel{K^c_2}{\leftarrow}B$,
and hence in $\Rel$ a product with the arrows reversed, which gives
projection relations $P_1 = K^c_1 = \set{((a,1),a) \mid a\in A}: A\uplus B \to A$
and likewise $P_2$.
}

\frame
  {   
    \frametitle{More examples of products}\label{Ch4:MorProdExa}

 \begin{itemize}[<+->]
\item NB in $\Rel$, $\emptyset$ is both initial and final, also since $\Rel\simeq\Rel^{op}$
\item Although we have given a proof on the previous slide, it is
instructive to verify the universal property with
$P_1 = \set{((a,1),a)\mid a\in A}$, $P_2 =  \set{((b,2),b)\mid b\in B}$
and $K=\set{(x,(a,1)) \mid R_1(x,a)}\cup \set{(x,(b,2)) \mid R_2(x,b)}$
\begin{tikzcd}[ampersand replacement=\&]
A \&
A\uplus B\arrow[swap]{l}{P_1} \arrow{r}{P_2}\&
B\\
\& X\arrow{ul}{R_1}\arrow{u}[description]{\exists!K} \arrow[swap]{ur}{R_2}
\end{tikzcd}
\item Products in the category $\Gra$ are as expected:
$G \stackrel{\pi_1}{\leftarrow} G\times H \stackrel{\pi_2}{\to}  H$,
where $\pi_1 = (\pi_{1,V},\pi_{1,E}) : G\times H \to G$ and $\pi_2: G\times H \to H$
are graph homomorphisms defined by the projections in $\Set$
\end{itemize}
 
}

\frame
  {   
    \frametitle{Products and sums in slice categories}\label{Ch4:ProdSumSlice}

 \begin{itemize}[<+->]
\item Let $\CC$ be a category with objects $A,B,T$with sum and product. 
Now (try to) consider in $\CC/T$:
\[
\begin{tikzcd}[ampersand replacement=\&]
A \arrow{r}{\kappa_1} \arrow[swap]{dr}{f}\&
A+B\arrow{d}[description]{[f,g]} \& 
B \arrow[swap]{l}{\kappa_2}\arrow{dl}{g} \&
A\arrow[swap]{dr}{f} \&
A\times B\arrow[swap]{l}{\pi_1} \arrow{r}{\pi_2}\arrow{d}{?}\&
B\arrow{dl}{g}\\
\& T\&\& \&
T %\arrow{ul}{f_1} \arrow[swap]{ur}{f_2}
\end{tikzcd}
\]
\item $\kappa_1: (A,f)\to(A+B,[f,ģ])$, $\kappa_2: (B,g)\to(A+B,[f,ģ])$ in $\CC/T$, 
and we can prove the universal property (exercise)
\item In the right diagram we are in trouble: 
take $A=B=\set{1,\pi}$ and $T=\set{\zet,\rea}$
and $f=g$ the typings $1:\zet, \pi: \rea$. 
How to type `mixed pairs'  $(1,\pi)$, $(\pi,1)$? Not, just throw them out!
%Impossible to do this compatible with the projections.
\item NB dualisation does give a product in the co-slice category
 \end{itemize}

 }

\frame
  {   
    \frametitle{Pullbacks and pushouts}\label{Ch4:PullPush}

 \begin{itemize}[<+->]
\item Pullbacks and pushouts generalize product and sum so that they become
compatible with (co-)slices. They are each other's dual and we do pullbacks first.
Fix a  category $\CC$.
\item A \emph{pullback} of a co-span
$A\stackrel{f}{\to} C \stackrel{g}{\leftarrow}B$
is a span $A\stackrel{\pi_1}{\leftarrow} D \stackrel{\pi_2}{\to}B$ such that 
$\pi_1;f=\pi_2;g$ and for any span 
$A\stackrel{h_1}{\leftarrow} X \stackrel{h_2}{\to}B$
with $h_1;f = h_2;g$
there is a unique morphism $k: X\to D$
such that $h_i = k;\pi_i$, i.e., % and $h_2 = k;\pi_2$. 
\begin{tikzcd}[ampersand replacement=\&]
X\arrow[swap]{ddr}{h_2}\arrow{rrd}{h_1}\arrow{rd}[description]{\exists!k}\\
\& 
D \arrow[swap]{r}{\pi_1}\arrow{d}{\pi_2}\&
A\arrow{d}{f}\\
\&
B\arrow{r}{g} \&
C 
\end{tikzcd} commutes

\item If such a pullback exists, object $D$ is denoted by $A\times_C B$ 
% (up to unique isomorphism) 
and the mediating morphism $k$ as $\langle f,g\rangle_C$

 \end{itemize}

 }

\frame
  {   
    \frametitle{Examples of pullbacks}\label{Ch4:Pullbexa}

 \begin{itemize}[<+->]
\item If $A\stackrel{f}{\to} C \stackrel{g}{\leftarrow}B$ in $\Set$,
then the span $A\stackrel{\pi_1}{\leftarrow} D \stackrel{\pi_2}{\to}B$ 
with $D=\set{(a,b) \in A\times B \mid f(a)=g(b) }$ is a pullback with
$k(x)=(h_1(x),h_2(x))$ in:
\begin{tikzcd}[ampersand replacement=\&]
X\arrow[swap]{ddr}{h_2}\arrow{rrd}{h_1}\arrow{rd}[description]{\exists!k}\\
\& 
D \arrow[swap]{r}{\pi_1}\arrow{d}{\pi_2}\&
A\arrow{d}{f}\\
\&
B\arrow{r}{g} \&
C 
\end{tikzcd}
\item NB $k$ is well-defined since $f(h_1(x))=g(h_2(x))$ so that $(h_1(x),h_2(x))\in D$,
 solving the `typing problem' on slide \ref{Ch4:ProdSumSlice}
\item Pullbacks in $\Gra$ are very much like those in $\Set$: 
a subgraph of the product graph. This is beautifully elaborated in the script.


 \end{itemize}

 }

\frame
  {   
    \frametitle{Monomorphisms characterized by pullback}\label{Ch4:PullbMono}

 \begin{itemize}[<+->]
\item Recall the definition of a monomorphism $f: A\to B$:
if
\begin{tikzcd}[ampersand replacement=\&]
%1 \arrow[yshift=0.5ex]{r}{y} \& 2, \arrow[yshift=-0.5ex]{l}{z}  \&
X \arrow[bend left]{r}{g}\arrow[bend right]{r}{h} \& 
A \arrow{r}{f}\& B
\end{tikzcd}
 commutes, then $g=h$ (for all $X,g,h$)

\item This we can say with pullbacks as well:
\[
\begin{tikzcd}[ampersand replacement=\&]
X\arrow[swap]{ddr}{h}\arrow{rrd}{g}\arrow{rd}[description]{\exists!k}\\
\& 
A \arrow[swap]{r}{id_A}\arrow{d}{id_A}\&
A\arrow{d}{f}\\
\&
A\arrow{r}{f} \&
B 
\end{tikzcd}
\]
\item So: $f$ is mono iff the pullback of 
$A\stackrel{f}{\to} B\stackrel{f}{\leftarrow}A$ is
$A\stackrel{id_A}{\leftarrow} A \stackrel{id_A}{\to}A$
 \end{itemize}

 }

\frame
  {   
    \frametitle{Kernels and graphs characterized by pullback}\label{Ch4:PullbKer}

 \begin{itemize}[<+->]
\item Recall the definition of the kernel of $f: A\to B$ in $\Set$:
$ker(f) = \set{(a,a') \in A\times A \mid f(a)=f(a')}$

\item This we can say with pullbacks as well (left diagram):
\[
\begin{tikzcd}[ampersand replacement=\&]
%X\arrow[swap]{ddr}{h_2}\arrow{rrd}{h_1}\arrow{rd}[description]{\exists!k}\\
\& 
ker(f) \arrow{r}{\pi_1}\arrow[swap]{d}{\pi_2}\&
A\arrow{d}{f}\& 
graph(f) \arrow{r}{\pi_1}\arrow[swap]{d}{\pi_2}\&
A\arrow{d}{f}\\
\&
A\arrow{r}{f} \&
B 
\&
B\arrow{r}{id_B} \&
B 
\end{tikzcd}
\]
\item So:  
$A\stackrel{\pi_1}{\leftarrow} ker(f) \stackrel{\pi_2}{\to}A$
is the pullback of 
$A\stackrel{f}{\to} B\stackrel{f}{\leftarrow}A$
\item Also, $graph(f) = \set{(a,b) \in A\times B \mid f(a)=b}$ (right diagram)
\item This generalises these notions from $\Set$ to arbitrary categories

 \end{itemize}

 }

\frame
  {   
    \frametitle{Properties of pullbacks}\label{Ch4:PullbProp}

 \begin{itemize}[<+->]
\item Pullbacks are unique up to unique isomorphism
\item If $\CC$ has 1 and pullbacks, then $A\times_1 B$ is a product
of $A,B$ 
\item Any slice category $\CC/T$ has pullbacks if $\CC$ has pullbacks
(well elaborated in the script)
\item In any slice category $\CC/T$ the object $(T,id_T)$ is final, so $\CC/T$
has products if $\CC$ has pullbacks
\item Pullbacks are also called \emph{fiber products}, products of preimages,
as in $\Set$ we have $A\times_C B = \cup_{c\in C} (f^-(c) \times g^-(c))$
(the last $C$ can be replaced by $f(A)\cap g(B)$)


 \end{itemize}

 }

\frame
  {   
    \frametitle{Pullback from product and equalizer}\label{Ch4:PullbProdEq}

 \begin{itemize}[<+->]
\item Given co-span $A\stackrel{f}{\to} C\stackrel{g}{\leftarrow}B$ in 
category $\CC$ with products, we get a co-span
$A\times B\stackrel{\pi_1;f}{\to} C\stackrel{\pi_2;g}{\leftarrow}A\times B$
of parallel morphisms
\item The pullback can now be obtained if $\CC$ also has equalizers:
\item Given two parallel morphisms $f,g: A\to B$, an \emph{equalizer}
consists of an object $E$ and a morphism $m:E\to A$ with $m;f=m;g$,
such that for all objects $X$ and morphisms $h:X\to A$ with $h;f=h;g$
there is a unique morphism $k: X\to E$ such that $k;m = h$. In a diagram:
\begin{tikzcd}[ampersand replacement=\&]
X\arrow{dr}{h}\arrow{d}[description]{\exists!k}\\ 
E \arrow[swap]{r}{m}\&
A\arrow[bend left]{r}{f}\arrow[bend right]{r}{g}\&B
\end{tikzcd}
In $\Set$, %\left\{
$\begin{array}{ll}E=\set{a\in A\mid f(a)=g(a)}\\m=in_{E,A}\end{array}$
\item The pullback $A\times_C B$ is then the equalizer of the parallel
morphisms $\pi_1;f$ and $\pi_2;g$ from $A\times B$ to $C$
 \end{itemize}

 }

\frame
  {   
    \frametitle{Preimages characterized by pullback}\label{Ch4:PullbPreim}

 \begin{itemize}[<+->]
\item Recall the preimage $f^-(b) = \set{a\in A \mid f(a)=b}$ of a map 
$f:A\to B$ and $b\in B$; for  a subset $B'\subseteq B$, 
the preimage $f^-(B') = \set{a\in A \mid f(a)\in B'}$ 
\item {\color{red} NB monad looming in the dark: $f^-(B')$ is the
`multiset image' of $B'$ under the multimap $f^-$, the flattened image!}
\item The preimage $f^-(B')$ can be characterised  as the pullback of the 
co-span $A\stackrel{f}{\to} B\stackrel{i}{\leftarrow}B'$ where $i=in_{B',B}$
is the inclusion map $B'\hookrightarrow B$:
%\[
\begin{tikzcd}[ampersand replacement=\&]
\set{(a,x) \mid a\in A, x\in B', f(a)=x} \arrow{r}{\pi_1}\arrow[swap]{d}{\pi_2}\&
A\arrow{d}{f}\\
B'\arrow{r}{i} \&
B 
\end{tikzcd}
%\]
\item Uhh, what's this $\set{(a,x) \mid a\in A, x\in B', f(a)=x}$, it's not $f^-(B')$?! OK,
an isomorphic span is $B'\stackrel{f}{\leftarrow} f^-(B') \stackrel{in}{\hookrightarrow}A$
\item Exercise: find the isomorphism for the isomorphic span
 \end{itemize}

 }

\frame
  {   
    \frametitle{The equalizer morphism is a mono}\label{Ch4:EqMorMono}

 \begin{itemize}[<+->]
\item Assume $(E,m)$ is an equalizer and assume $h: X\to E$:
\[
\begin{tikzcd}[ampersand replacement=\&]
X\arrow{dr}{h;m}\arrow[swap]{d}{h}\\ 
E \arrow[swap]{r}{m}\&
A\arrow[bend left]{r}{f}\arrow[bend right]{r}{g}\&B
\end{tikzcd}
\]
\item Then $(h;m);f = h;(m;f) = h;(m;g) = (h;m);g$,
so $h$ is uniquely defined by $h;m$ by the universal property of  $(E,m)$
\item So if  $h;m = h';m$ for $h':X\to E$ we have $h=h'$
\item Hence $m$ is mono
 \end{itemize}

 }

\frame
  {   
    \frametitle{Pullbacks reflect monos}\label{Ch4:PullReflMono}

 \begin{itemize}[<+->]
\item Assume the square below is a pullback square, $m:B\to C$ a mono,
and $g,h: X\to D$:
\begin{tikzcd}[ampersand replacement=\&]
X\arrow[bend left]{dr}{h}\arrow[bend right]{dr}{g}\\
\& 
D \arrow[swap]{r}{\pi_1}\arrow{d}{\pi_2}\&
A\arrow{d}{f}\\
\&
B\arrow{r}{m} \&
C 
\end{tikzcd}
\item We prove that $\pi_1$ is mono: assume $g;\pi_1 = h;\pi_1$.
Then $g;\pi_1;f = h;\pi_1;f$, so $g;\pi_2;m = h;\pi_2;m$ by commutativity.
So also $g;\pi_1 = h;\pi_1$, since $m$ is mono. Hence $g=h$ is the
unique mediating morphism.
\item Exercise: what if $f$ is mono instead of $m$? Why?

\end{itemize}

 }

\frame
  {   
    \frametitle{Equalizer from pullbacks and product}\label{Ch4:EqByPullbProd}

 \begin{itemize}[<+->]
\item Given in a category $\CC$:
\begin{tikzcd}[ampersand replacement=\&]
%X\arrow{dr}{h;m}\arrow[swap]{d}{h}\\ 
E \arrow[swap]{r}{m}\&
A\arrow[bend left]{r}{f}\arrow[bend right]{r}{g}\&B
\end{tikzcd}
\item $(E,m)$ is an equalizer iff the following is a pullback square:
\[
\begin{tikzcd}[ampersand replacement=\&]
%X\arrow[bend left]{dr}{h}\arrow[bend right]{dr}{g}\\
E \arrow{r}{m;f}\arrow[swap]{d}{m}\&
B\arrow{d}{\Kp{id_B,id_B}}\\
A\arrow{r}{\Kp{f,g}} \& B\times B 
\end{tikzcd}
\]
\item Recall that products can be formed by pullbacks $A\to 1_\CC \leftarrow B$
\item Is there something more general?
\item Yes: limits

\end{itemize}

 }


\frame
  {   
    \frametitle{Summary Chapter 4}\label{Ch4:Summary}

 \begin{enumerate}[<+->]
\item 
 \end{enumerate}


}

\frame
  {   
    \frametitle{Comments on script}\label{Ch4:comments}

 \begin{itemize}[<+->]
\item In formula (4.22) and the para above, the $h$'s should be $f_x$'s
\item In the diagram of Cor. 4.4.110, the left lower $A$ should be $B$
 \end{itemize}

 }

\end{document}

\myurl{en.wikipedia.org/wiki/Ordered_pair}
 \begin{itemize}
    \item emulation halts because of the partiality of $\delta$;
    \item emulation halts because of "head left" with head at first cell;
    \item emulation halts because of reaching a halting state;
    \item the emulated TM goes on forever.
 \end{itemize}
