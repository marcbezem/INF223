\documentclass[handout]{beamer}
%\documentclass[slides]{beamer}
% Vary the color applet  (try out your own if you like)
%\colorlet{structure}{red!20!black}
%\beamertemplateshadingbackground{yellow!20}{white}
%\usepackage{beamerthemeshadow}
%\usepackage[utf8x]{inputenc} CONFLICT!
%\usepackage[english,norsk,nynorsk]{babel}
\usepackage{tikz}
\usepackage{tikz-cd} 
\usetikzlibrary{trees}

\usepackage[all]{xy}
\usepackage{multicol}

%\setbeamertemplate{navigation symbols}{}++++++
%\setbeamertemplate{footline}[frame number]
\usetheme{Montpellier}
\setbeamertemplate{footline}[frame number]


\input macros

\newcommand{\To}{\Rightarrow}
\newcommand{\Trt}{\stackrel{*}{\Rightarrow}}
\newcommand{\ToG}{\Rightarrow_G}
\newcommand{\redS}{{\color{red} S}}

\newcommand{\bfsf}[1]{{\boldsymbol{#1}}}
\newcommand{\Set}{\bfsf{Set}}
\newcommand{\Gra}{\bfsf{Graph}}
\newcommand{\CC}{\bfsf{C}}
\newcommand{\DD}{\bfsf{D}}
\newcommand{\EE}{\bfsf{E}}
\newcommand{\PP}{\bfsf{P}}
\newcommand{\Nat}{\bfsf{Nat}}
\newcommand{\Incl}{\bfsf{Incl}}
\newcommand{\Rel}{\bfsf{Rel}}
\newcommand{\Mult}{\bfsf{Mult}}
\newcommand{\Mon}{\bfsf{Mon}}
\newcommand{\Cat}{\bfsf{Cat}}
\newcommand{\CAT}{\bfsf{CAT}}


\title[INF223 presentations]{}

\begin{document}

\section{Chapter 4}
\subsection{Properties and Constructions}
 
\frame
  {   
    \frametitle{Equivalences}\label{Ch4:Eqvs}

 \begin{itemize}[<+->]
\item One method of abstraction: make equality coarser, by making
some elements `equal' that were not equal before
\item Such a new `equality' called \emph{equivalence} should have some important 
properties that any equality relations has: reflexivity, symmetry and transitivity 
(\emph{refsymtra})
\item There are at least 3 ways to define an equivalence on a set $A$:
   \begin{enumerate}[<+->]
\item directly, by defining a refsymtra relation ${\simeq}\subseteq A\times A$
\item by partitioning $A = \cup_{i\in I} A_i $ in non-empty, disjoint, I-indexed $A_i$
\item by defining a (typing) map $f: A\to T$ to some set $T$
   \end{enumerate}
\item These three ways are logically equivalent ($\approx$ MNF130): 
   \begin{enumerate}[<+->]
\item $[a]_\simeq = \set{x\in A\mid a\simeq x}$, 
$A/{\simeq}=\set{[a]_\simeq \mid a\in A}$, $[\_] : A \to A/{\simeq}$
\item $a\simeq a' $ iff $a,a' \in A_i$ is refsymtra, and $I = A/{\simeq}$, $A_i = i$
\item $a\simeq a' $ iff $f(a)=f(a')$ is refsymtra, and  $T = A/{\simeq}$, $f=[\_]$
   \end{enumerate}
\item If $f$ in 3 is surjective, the equivalence classes are the preimages
 \end{itemize}

 }

\frame
  {   
    \frametitle{Example: equality modulo}\label{Ch4:EqExa}

 \begin{itemize}[<+->]
\item Let $\zet$ be the set of integers and $k\in\zet$, $k>0$
\item We define $=_k$, equality modulo $k$ :
   \begin{itemize}[<+->]
\item directly, $x =_k y$ iff $x-y$ is a multiple of $k$
\item partitioning $\zet = \cup_{i\in \set{0,...,k-1}} \zet_i $, $\zet_i = \set{kz+i \mid z\in\zet}$
\item by defining  $m_k: \zet\to \set{0,...,k-1}$, $m_k(z) = (z \mod k)$ 
   \end{itemize}
\item Q: which operations/relations carry over from $\zet$ to $\zet/{=_k}$?
\item A: only those that are \emph{invariant} under ${=_k}$, i.e., are
independent of the choice of representative:
   \begin{itemize}[<+->]
\item for function $f$:  $f(z) = f(z')$ if $z =_k z'$ 
\item for relation $R$:  $R(z,u) \iff R(z',u')$    if $z =_k z'$ and $u =_k u'$
\item similarly for other arities
   \end{itemize}
\item Example: \emph{modulo arithmetic} (addition, multiplication)
\item {\color{red}NB}: $2^0 = 1 \neq_3 8 = 2^3$, whereas $0 =_3 3$
 \end{itemize}

 }

\frame
  {   
    \frametitle{More examples}\label{Ch4:EqMorExa}

 \begin{itemize}[<+->]
\item Logical equivalence
\item Isomorphy of objects of a category $\CC$ is a equivalence
\item Isomorphy of categories is an equivalence
\item Every category is isomorphic to a quotient of the path category of
its underlying graph 
\item Equivalence of categories is an even coarser equivalence
 \end{itemize}

 }



\frame
  {   
    \frametitle{Summary Chapter 4}\label{Ch4:Summary}

 \begin{enumerate}[<+->]
\item 
 \end{enumerate}


}

\frame
  {   
    \frametitle{Comments on script}\label{Ch4:comments}

 \begin{itemize}[<+->]
\item 
 \end{itemize}

 }

\end{document}

\myurl{en.wikipedia.org/wiki/Ordered_pair}
 \begin{itemize}
    \item emulation halts because of the partiality of $\delta$;
    \item emulation halts because of "head left" with head at first cell;
    \item emulation halts because of reaching a halting state;
    \item the emulated TM goes on forever.
 \end{itemize}
