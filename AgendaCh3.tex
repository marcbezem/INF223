\documentclass[handout]{beamer}
%\documentclass[slides]{beamer}
% Vary the color applet  (try out your own if you like)
%\colorlet{structure}{red!20!black}
%\beamertemplateshadingbackground{yellow!20}{white}
%\usepackage{beamerthemeshadow}
%\usepackage[utf8x]{inputenc} CONFLICT!
\usepackage{tikz}
%\usepackage[english,norsk,nynorsk]{babel}
\usepackage{tikz-cd}
\usetikzlibrary{trees}

\usepackage[all]{xy}
\usepackage{multicol}

%\setbeamertemplate{navigation symbols}{}++++++
%\setbeamertemplate{footline}[frame number]
\usetheme{Montpellier}


\input macros

\newcommand{\To}{\Rightarrow}
\newcommand{\Trt}{\stackrel{*}{\Rightarrow}}
\newcommand{\ToG}{\Rightarrow_G}
\newcommand{\redS}{{\color{red} S}}

\newcommand{\bfsf}[1]{{\boldsymbol{#1}}}
\newcommand{\Set}{\bfsf{Set}}
\newcommand{\Gra}{\bfsf{Graph}}
\newcommand{\CC}{\bfsf{C}}
\newcommand{\DD}{\bfsf{D}}
\newcommand{\EE}{\bfsf{E}}
\newcommand{\PP}{\bfsf{P}}
\newcommand{\HH}{\bfsf{H}}
\newcommand{\Nat}{\bfsf{Nat}}
\newcommand{\Incl}{\bfsf{Incl}}
\newcommand{\Rel}{\bfsf{Rel}}
\newcommand{\Mult}{\bfsf{Mult}}
\newcommand{\Mon}{\bfsf{Mon}}
\newcommand{\Cat}{\bfsf{Cat}}
\newcommand{\CAT}{\bfsf{CAT}}
\title[INF223 presentations]{}

\begin{document}

\section{Chapter 3}
\subsection{Natural transformations}
 
\frame
  {   
    \frametitle{Natural Transformations}\label{Ch3:NatTrans}

 \begin{itemize}[<+->]
\item Cold turkey: given functors $F,G: \CC\to\DD$, a \emph{natural transformation}
$\alpha: F\To G$ maps every object $A$ of $\CC$ to a morphism $\alpha_A : F(A)\to G(A)$
in $\DD$ such that, for every $f: A\to B$ in $\CC$ the following
\emph{naturality diagram} commutes:
\[
\begin{tikzcd}[ampersand replacement=\&]
A \arrow[swap]{d}{f} \& F(A)  \arrow[swap]{d}{F(f)}\arrow{r}{\alpha_A} \&\arrow{d}{G(f)} G(A)\\
B \& F(B)  \arrow{r}{\alpha_B} \& G(B)
\end{tikzcd}
\]
\item In $\HH$, Haskell-as-a-category, with types as objects and functions as morphisms:
   \begin{itemize}[<+->]
\item $[\_]_V$ maps any type $T$ to type $[T]$ of lists over $T$
\item $[\_]_E$ maps any $f: T\to T'$ to $map~f: [T] \to [T']$
\item $[\_]:\HH\to\HH$ is a functor, $reverse: [\_] \To [\_]$ is a natural transformation
   \end{itemize}
 \end{itemize}

 }

\frame
  {   
    \frametitle{What is natural about reversing lists?}\label{Ch3:reverse}

 \begin{itemize}[<+->]
\item Why is $reverse: [\_] \To [\_]$ a natural transformation?\\
Because this diagram commutes:
\[
\begin{tikzcd}[ampersand replacement=\&]
A \arrow[swap]{d}{f} \& {[A]}  \arrow[swap]{d}{map~f}\arrow{rr}{reverse_A} \&\& {[A]}\arrow{d}{map~f}\\
B \& {[B]}  \arrow{rr}{reverse_B}\& \& {[B]}
\end{tikzcd}
\]
\item What is natural about reversing lists?
That you can program this without knowing what the type of the elements is.
In other cultures, this is called \emph{generic programming},
or \emph{type polymorphism}
\item NB what naturality means depends on the categories $\CC,\DD$
 \end{itemize}

 }

\frame
  {   
    \frametitle{Yet another natural transformation}\label{Ch3:length}

 \begin{itemize}[<+->]
\item Functors can be constant
\item Example in $\HH$: the functor $C(\nat)$ 
   \begin{itemize}[<+->]
\item $C(\nat)_V$ maps any type $T$ to type $\nat$
\item $C(\nat)_E$ maps any $f: T\to T'$ to $id_\nat: C(\nat)(T) \to C(\nat)(T')$
   \end{itemize}
\item $[\_],C(\nat):\HH\to\HH$ are functors, $length: [\_] \To C(\nat)$ 
is a natural transformation. Why? \vspace*{-0.5cm}
\item Because this diagram commutes:
\begin{tikzcd}[ampersand replacement=\&]
A \arrow[swap]{r}{f} \& B\\
{[A]}  \arrow[swap]{d}{length_A}\arrow{r}{map~f} \&{[B]}  \arrow{d}{length_B}\\
\nat \arrow{r}{id_\nat} \& \nat
\end{tikzcd}
\item What is natural about computing the length of a list?\\
That you can program $length_A$ uniformly in $A$
 \end{itemize}

 }


\frame
  {   
    \frametitle{Metamodel of graphs}\label{Ch3:MG}

 \begin{itemize}[<+->]
\item Small graphs are given by two sets and two functions
\item More precisely: if the sets are $Nod,Arr$ we have $beg,end: Nod\to Arr$
\item This data has itself the form of a graph, a metamodel!
\item Define $MG = 
\begin{tikzcd}[ampersand replacement=\&]
Arr \arrow[yshift=0.5ex]{r}{beg} \arrow[swap, yshift=-0.5ex]{r}{end} \& Nod,
\end{tikzcd} $ or:
   \begin{itemize}[<+->]
\item $MG_V = \set{Nod,Arr}$
\item $MG_E = \set{beg,end}$
\item $sc^{MG}(beg) = sc^{MG}(end) = Arr$
\item $tg^{MG}(beg) = tg^{MG}(end) = Nod$
   \end{itemize}
\item Small graphs are now graph homomorphisms $MG \to gr(\Set)$
\item Graph homomorphisms are now natural transformations (...)
 \end{itemize}

 }


\frame
  {   
    \frametitle{Comments on script}\label{Ch3:comments}

 \begin{itemize}[<+->]
\item 
 \end{itemize}

 }



\end{document}

\myurl{en.wikipedia.org/wiki/Ordered_pair}
 \begin{itemize}
    \item emulation halts because of the partiality of $\delta$;
    \item emulation halts because of "head left" with head at first cell;
    \item emulation halts because of reaching a halting state;
    \item the emulated TM goes on forever.
 \end{itemize}
