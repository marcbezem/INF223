\documentclass[handout]{beamer}
%\documentclass[slides]{beamer}
% Vary the color applet  (try out your own if you like)
%\colorlet{structure}{red!20!black}
%\beamertemplateshadingbackground{yellow!20}{white}
%\usepackage{beamerthemeshadow}
%\usepackage[utf8x]{inputenc} CONFLICT!
\usepackage{tikz}
%\usepackage[english,norsk,nynorsk]{babel}
\usepackage{tikz-cd}
\usetikzlibrary{trees}

\usepackage[all]{xy}
\usepackage{multicol}

%\setbeamertemplate{navigation symbols}{}++++++
%\setbeamertemplate{footline}[frame number]
\usetheme{Montpellier}


\input macros

\newcommand{\To}{\Rightarrow}
\newcommand{\Trt}{\stackrel{*}{\Rightarrow}}
\newcommand{\ToG}{\Rightarrow_G}
\newcommand{\redS}{{\color{red} S}}

\newcommand{\bfsf}[1]{{\boldsymbol{#1}}}
\newcommand{\Set}{\bfsf{Set}}
\newcommand{\Gra}{\bfsf{Graph}}
\newcommand{\CC}{\bfsf{C}}
\newcommand{\DD}{\bfsf{D}}
\newcommand{\EE}{\bfsf{E}}
\newcommand{\PP}{\bfsf{P}}
\newcommand{\HH}{\bfsf{H}}
\newcommand{\Nat}{\bfsf{Nat}}
\newcommand{\Incl}{\bfsf{Incl}}
\newcommand{\Rel}{\bfsf{Rel}}
\newcommand{\Mult}{\bfsf{Mult}}
\newcommand{\Mon}{\bfsf{Mon}}
\newcommand{\Cat}{\bfsf{Cat}}
\newcommand{\CAT}{\bfsf{CAT}}
\title[INF223 presentations]{}

\begin{document}

\section{Chapter 3}
\subsection{3.1 Natural transformations}
 
\frame
  {   
    \frametitle{Natural Transformations}\label{Ch3:NatTrans}

 \begin{itemize}[<+->]
\item Cold turkey: given functors $F,G: \CC\to\DD$, a \emph{natural transformation}
$\alpha: F\To G$ maps every object $A$ of $\CC$ to a morphism $\alpha_A : F(A)\to G(A)$
in $\DD$ such that, for every $f: A\to B$ in $\CC$ the following
\emph{naturality diagram} commutes:
\[
\begin{tikzcd}[ampersand replacement=\&]
A \arrow[swap]{d}{f} \& F(A)  \arrow[swap]{d}{F(f)}\arrow{r}{\alpha_A} \&\arrow{d}{G(f)} G(A)\\
B \& F(B)  \arrow{r}{\alpha_B} \& G(B)
\end{tikzcd}
\]
\item In $\HH$, Haskell-as-a-category, with types as objects and functions as morphisms:
   \begin{itemize}[<+->]
\item $[\_]_V$ maps any type $T$ to type $[T]$ of lists over $T$
\item $[\_]_E$ maps any $f: T\to T'$ to $map~f: [T] \to [T']$
\item $[\_]:\HH\to\HH$ is a functor, $reverse: [\_] \To [\_]$ is a natural transformation
   \end{itemize}
 \end{itemize}

 }

\frame
  {   
    \frametitle{What is natural about reversing lists?}\label{Ch3:reverse}

 \begin{itemize}[<+->]
\item Why is $reverse: [\_] \To [\_]$ a natural transformation?\\
Because this diagram commutes:
\[
\begin{tikzcd}[ampersand replacement=\&]
A \arrow[swap]{d}{f} \& {[A]}  \arrow[swap]{d}{map~f}\arrow{rr}{reverse_A} \&\& {[A]}\arrow{d}{map~f}\\
B \& {[B]}  \arrow{rr}{reverse_B}\& \& {[B]}
\end{tikzcd}
\]
\item What is natural about reversing lists?
That you can program this without knowing what the type of the elements is.
In other cultures, this is called \emph{generic programming},
or \emph{type polymorphism}
\item NB what naturality means depends on the categories $\CC,\DD$
 \end{itemize}

 }

\frame
  {   
    \frametitle{Yet another natural transformation}\label{Ch3:length}

 \begin{itemize}[<+->]
\item Functors can be constant
\item Example in $\HH$: the functor $C(\nat)$ 
   \begin{itemize}[<+->]
\item $C(\nat)_V$ maps any type $T$ to type $\nat$, $C(\nat)_V(T)$
\item $C(\nat)_E$ maps any $f: T\to T'$ to $id_\nat: \nat\to\nat$
   \end{itemize}
\item $[\_],C(\nat):\HH\to\HH$ are functors (why?), $length: [\_] \To C(\nat)$ 
is a natural transformation. Why? \vspace*{-0.5cm}
\item Because this diagram commutes:
\begin{tikzcd}[ampersand replacement=\&]
A \arrow[swap]{r}{f} \& B\\
{[A]}  \arrow[swap]{d}{length_A}\arrow{r}{map~f} \&{[B]}  \arrow{d}{length_B}\\
\nat \arrow{r}{id_\nat} \& \nat
\end{tikzcd}
\item What is natural about computing the length of a list?\\
That you can program $length_A$ uniformly in $A$.
 \end{itemize}
 
}

\frame
  {   
    \frametitle{An unnatural transformation}\label{Ch3:unnat}

 \begin{itemize}[<+->]
\item Is removing duplicates $nodup_A : [A] \to [A]$ natural in $A$?
\item No, this diagram need not commute:
\[
\begin{tikzcd}[ampersand replacement=\&]
A \arrow[swap]{d}{f} \& {[A]}  \arrow[swap]{d}{map~f}\arrow{rr}{nodup_A} \&\& {[A]}\arrow{d}{map~f}\\
B \& {[B]}  \arrow{rr}{nodup_B}\& \& {[B]}
\end{tikzcd}
\]
\item Why not?
\item Fix: codomain is the wrong category, take $[\_],\powset : \Set\to\Set$ as functors
and it is easy to define a natural $list2set : [\_]\To\powset$,
$\left\{\begin{array}{ll}list2set_A([]) &= \emptyset\\list2set_A(a:as) &=\set{a}\cup list2set_A(as)
\end{array}\right.$
 \end{itemize}

 }

\frame
  {   
    \frametitle{Sorting, naturally}\label{Ch3:sorting}

 \begin{itemize}[<+->]
\item For sorting we need to equip the type $A$ with an ordering $\leq$
\item This changes the category to $\Set^{<}$: objects are ordered sets (or types) and
morphisms are order-preserving maps, i.e., $f: (A,\leq)\to (B,\preceq)$ 
such that $f(a)\preceq f(a')$ for all $a\leq a'$. 
\item Then we should also order $[A]$! Take $\leq_{lex}$, and observe:
if $l \leq_{lex} l'$, then $map~f~l \preceq_{lex} map~f~l'$ for any $f$ as above.
\item This shows that we have a functor $[\_]^{<} : \Set^{<} \to \Set^{<}$
   \begin{itemize}[<+->]
\item mapping object $(A,\leq)$ to object $([A],\leq_{lex})$, 
\item just $map$ on the morphisms of $\Set^{<}$ (order-preserving $f$'s).
   \end{itemize}
\item Sorting before or after mapping $f$ makes no difference:
\[
\begin{tikzcd}[ampersand replacement=\&]
(A,\leq) \arrow[swap]{d}{f} \& {[A]}  \arrow[swap]{d}{map~f}\arrow{rr}{sort_A} \&\& {[A]}\arrow{d}{map~f}\\
(B,\preceq) \& {[B]}  \arrow{rr}{sort_B}\& \& {[B]}
\end{tikzcd}
\]
 \end{itemize}

 }


\frame
  {   
    \frametitle{A hierarchy of (meta)models}\label{Ch3:metameta}

 \begin{itemize}[<+->]
\item A formal definition of a grammar (metametametamodel), e.g.,
$(\mathcal T, \mathcal N, S, \mathcal R)$ with $S\in\mathcal N$ and
$\mathcal R \subseteq(\mathcal N \cup\mathcal T)^* \times (\mathcal N \cup\mathcal T)^*$, ...
\item A formal definition of the grammar class (metametamodel), e.g., CFGs
$(\mathcal T, \mathcal N, S, \mathcal R)$ with $S\in\mathcal N$ and
$\mathcal R \subseteq\mathcal N \times (\mathcal N \cup\mathcal T)^*$
\item A specific context-free grammar describing the syntax (metamodel)
\item A program written in a this specific syntax (model)
\item A machine performing a specific task (or not, that's reality :-)
 \end{itemize}

 }

\frame
  {   
    \frametitle{Metamodel of graphs}\label{Ch3:MG}

 \begin{itemize}[<+->]
\item Small graphs are given by two sets and two functions of a specific type,
if the sets are called $Nod,Arr$, then the functions can be called $beg,end: Arr\to Nod$
\item This data has itself the form of a graph, a metamodel!
\item Define the metagraph $MG = 
\begin{tikzcd}[ampersand replacement=\&]
Arr \arrow[yshift=0.5ex]{r}{beg} \arrow[swap, yshift=-0.5ex]{r}{end} \& Nod,
\end{tikzcd} $ or:
   \begin{itemize}[<+->]
\item $MG_V = \set{Nod,Arr}$
\item $MG_E = \set{beg,end}$
\item $sc^{MG}(beg) = sc^{MG}(end) = Arr$
\item $tg^{MG}(beg) = tg^{MG}(end) = Nod$
   \end{itemize}
\item Small graphs are now graph homomorphisms $MG \to gr(\Set)$
\item Graph homomorphisms are now natural transformations (...)
 \end{itemize}

 }

\frame
  {   
    \frametitle{Exploring the idea of the metagraph MG}\label{Ch3:ideaMG}

 \begin{itemize}[<+->]
\item As graph \begin{tikzcd}[ampersand replacement=\&]
Arr \arrow[yshift=0.5ex]{r}{beg} \arrow[swap, yshift=-0.5ex]{r}{end} \& Nod,
\end{tikzcd} since there are no composable arrows,  $MG$ is almost its path category:
 just add identities  %($= []_{Arr}, []_{Nod}$)
\item Similarly, any graph homomorphism $MG\to gr(\CC)$ is almost a functor to
the category $\CC$: just map identities to identities
\item Generally: extend $\varphi: G\to gr(\CC)$ to functor $\varphi^*: \PP(G)\to\CC$
\item A graph homomorphism $F : MG\to gr(\Set)$ consists of
   \begin{itemize}[<+->]
\item a set $F_V(Nod)$ the we call the set of nodes
\item a set $F_V(Arr)$ the we call the set of arrows
\item a map $F_E(beg) : F_V(Arr)\to F_V(Nod)$, the source map $sc$
\item a map $F_E(end) : F_V(Arr)\to F_V(Nod)$, the target map $tg$
   \end{itemize}
and in this way you can get any small graph by a suitable $F$!
\item We call such $F$'s \emph{interpretations} of $MG$ % , also written $F : MG\to \Set$
 \end{itemize}

 }

\frame
  {   
    \frametitle{Interpretations  of a graph}\label{Ch3:interpretations}

 \begin{itemize}[<+->]
\item An \emph{interpretation} of a graph $M$ in a category $\CC$
is a graph homomorphism $F: M\to gr(\CC)$ also written $F : M\to \CC$
\item Example: small hypergraphs, with arrows  between two \emph{sets} of nodes,
are interpretations $MG\to \Mult$ 
\item A graph homomorphism $F : MG\to gr(\Mult)$ consists of
   \begin{itemize}[<+->]
\item a set $F_V(Nod)$ the we call the set of nodes
\item a set $F_V(Arr)$ the we call the set of arrows
\item a {\color{red}multi}map $F_E(beg) : F_V(Arr)\to F_V(Nod)$, the source map:  
$F_E(beg)(a)\in {\color{red}\powset} F_V(Nod)$ for all $a\in F_V(Arr)$
\item a {\color{red}multi}map $F_E(end) : F_V(Arr)\to F_V(Nod)$, the target map: 
 $F_E(end)(a)\in {\color{red}\powset} F_V(Nod)$  for all $a\in F_V(Arr)$,
   \end{itemize}
\item In this way you can get any small {\color{red}hyper}graph by a suitable $F$!
\item By varying the category $\CC$ we get lots of interpretations
 \end{itemize}

 }

\frame
  {   
    \frametitle{Homomorphisms recovered (1)}\label{Ch3:Homs1}

 \begin{itemize}[<+->]
\item With the {interpretations} of the graph $MG$ in a the category $\Set$
being small graphs, where are the graph homomorphisms?
\item Let $M$ be a graph and $F,G : M\to \CC$ two interpretations. They can be promoted
to functors $F^*,G^* : \PP(M)\to \CC$. Any natural transformation
$\alpha: F^* \To G^*$ satisfies
\[
\begin{tikzcd}[ampersand replacement=\&]
v \arrow[swap]{d}{[f_1,...,f_n]} \&\& {F^*(v)}  \arrow[swap]{d}{F^*([f_1,...,f_n])}\arrow{rr}{\alpha_v} 
\&\& {G^*(v})\arrow{d}{G^*([f_1,...,f_n])}\\
u \&\&{F^*(u)}  \arrow{rr}{\alpha_u}\& \& {G^*(u)}
\end{tikzcd}
\]
where $F^*([f_1,...,f_n]) = F(f_1);...;F(f_n)$ and likewise for $G$
\item This will be simplified from $\PP(G)$ to $G$ in the next slide
 \end{itemize}

 }

\frame
  {   
    \frametitle{Homomorphisms recovered (2)}\label{Ch3:Homs2}

 \begin{itemize}[<+->]
\item Recall we have {interpretations} $F,G : M\to \CC$ and a natural transformation
$\alpha: F^* \To G^*$. By list recursion the following
naturality diagram is sufficient for the previous one:
\[
\begin{tikzcd}[ampersand replacement=\&]
v \arrow[swap]{d}{f} \&\& {F(v)}  \arrow[swap]{d}{F(f)}\arrow{rr}{\alpha_v} 
\&\& {G(v})\arrow{d}{G(f)}\\
u \&\&{F(u)}  \arrow{rr}{\alpha_u}\& \& {G(u)}
\end{tikzcd}
\]
\item This is about data in $F,G : M\to \CC$ only and the script calls 
such $\alpha$ a natural transformation of interpretations
\item The next slide recovers graph homomorphisms as
the special case $M=MG$ and $\CC=\Set$, but the generality here is useful
 \end{itemize}

 }

\frame
  {   
    \frametitle{Homomorphisms recovered (3)}\label{Ch3:Homs3}

 \begin{itemize}[<+->]
\item $MG = 
\begin{tikzcd}[ampersand replacement=\&]
Arr \arrow[yshift=0.5ex]{r}{beg} \arrow[swap, yshift=-0.5ex]{r}{end} \& Nod,
\end{tikzcd} $ $F,G: MG\to\Set$, $\alpha:F\To G$
\[
\begin{tikzcd}[ampersand replacement=\&]
Arr \arrow[swap]{d}{beg} \&\& {F(Arr)}  \arrow[swap]{d}{F(beg)}\arrow{rr}{\alpha_{Arr}} 
\&\& {G(Arr})\arrow{d}{G(beg)}\\
Nod \&\&{F(Nod)}  \arrow{rr}{\alpha_{Nod}}\& \& {G(Nod)}
\end{tikzcd}
\]
+
\[
\begin{tikzcd}[ampersand replacement=\&]
Arr \arrow[swap]{d}{end} \&\& {F(Arr)}  \arrow[swap]{d}{F(end)}\arrow{rr}{\alpha_{Arr}} 
\&\& {G(Arr})\arrow{d}{G(end)}\\
Nod \&\&{F(Nod)}  \arrow{rr}{\alpha_{Nod}}\& \& {G(Nod)}
\end{tikzcd}
\]
= graph homomorphism represented by $\alpha$ 
 \end{itemize}

 }


\frame
  {   
    \frametitle{Opposite functors and natural transformations}\label{Ch3:oppnat}

 \begin{itemize}[<+->]
\item Any functor $F:\CC\to\DD$ has an opposite $F^{op}:\CC^{op}\to\DD^{op}$,
same on objects,  and $F^{op}(f^{op}) = F(f)^{op}$ for all $f\in \CC(A,B)$
\item For natural transformations, taking opposites does this:
%$F,G:\CC\to\DD$ and $\alpha:F\To G$
\[
\begin{tikzcd}[ampersand replacement=\&]
A \arrow[swap]{d}{f} \& F(A)   \arrow{r}{\alpha_A}\arrow[swap]{d}{F(f)}\&\arrow[swap]{d}{G(f)} G(A)
\& A \& F(A)   \& \arrow[swap]{l}{(\alpha_A)^{op}} G(A)
\\
B \& F(B)  \arrow{r}{\alpha_B} \& G(B)
\&B  \arrow{u}{f^{op}} \& F(B) \arrow{u}{F(f)^{op}}  \&\arrow{u}{G(f)^{op}} \color{red} \arrow[swap]{l}{(\alpha_B)^{op}}G(B)
\end{tikzcd}
\]
\item Naturality left (in $\DD$) should imply naturality right (in $\DD^{op})$. 
It does, starting in $\color{red}G(B)$! 
%but as a natural transformation $G^{op}\To F^{op}$, 
\item Define: $(\alpha^{op})_A = (\alpha_A)^{op}$ to get $\alpha^{op}: G^{op}\To F^{op}$
\item The opposite natural transformation $\alpha^{op}$ goes the other way
 \end{itemize}

 }

\frame
  {   
    \frametitle{Functor categories}\label{Ch3:functorcats}

 \begin{itemize}[<+->]
\item For categories $\CC,\DD$ and functor $F: \CC\to\DD$, the 
\emph{identity transformation} $id_F : F\To F$ is defined by $(id_F)_A = id^\DD_{F(A)}$
\item For functors $F,G,H: \CC\to\DD$, and natural
transformations $\alpha: F\To G$ and $\beta:G\To H$, the composition
$\alpha;\beta: F\To H$ is a natural transformation defined by 
$(\alpha;\beta)_A= \alpha_A;\beta_A$
 \item The \emph{functor category} $[\CC\to\DD]$ has:
   \begin{itemize}[<+->]
\item all functors $\CC\to\DD$ as objects 
\item all natural transformations between functors as morphisms 
\item identities and composition as above, satisfying ASS and ID
   \end{itemize}
\item * The category $\Cat$ can now be made into a 
\myurl{https://ncatlab.org/nlab/show/2-category} with small categories 
as objects, 1-morphisms as functors, and natural transformations as 2-morphisms
 \end{itemize}

 }

\frame
  {   
    \frametitle{Interpretation categories}\label{Ch3:interpretationcats}

 \begin{itemize}[<+->]
\item For any graph $M$,  category $\DD$ and interpretation $G: M\to\DD$, the 
\emph{identity transformation} $id_G : G\To G$ sends $v$ to $id^\DD_{G(v)}$
\item For interpretations $F,G,H: M\to\DD$, and natural
transforma- tions $\alpha: F\To G$ and $\beta:G\To H$, the composition
$\alpha;\beta: F\To H$ sends $v$ to $\alpha_v;\beta_v$, well defined and natural
 \item The \emph{interpretation category} $[M\to\DD]$ has:
   \begin{itemize}[<+->]
\item all interpretations $M\to\DD$ as objects 
\item all natural transf.s  between interpretations as morphisms 
\item identities and composition as above, satisfying ASS and ID
   \end{itemize}
\item The above notions can also be obtained from the corresponding
notions for functor categories by extending interpretations
to functors on the path category
 \end{itemize}

 }

\subsection{3.2 Indexing}

\frame
  {   
    \frametitle{Indexing with sets}\label{Ch3:indexset}

 \begin{itemize}[<+->]
\item Given a set $I$, an $I$-indexed family of sets is a map $A$
sending any $i\in I$ to a set denoted $A_i$. We write $A=(A_i \mid i\in I)$
\item We can promote any set $I$ to a graph without edges: 
$(I,\emptyset,in_{\emptyset,I},in_{\emptyset,I})$, also denoted as $I$ (mild abuse of notation)
 \item $I$-indexed families of sets are now interpretations in $[I\to \Set]$
 \item A morphism $f: A\to B$ in $[I\to \Set]$ is then a natural transformation
such that $f_i: A_i \to B_i$ in $\Set$, a family of maps that we also denote
$(f_i \mid i\in I)$ or $(f_i: A_i\to B_i \mid i\in I)$
\item No worries about naturality, in the absence of arrows
\item Indexing is common: records, 
vector spaces $R^i$ ($I=\nat$), matrices $R^{n,m}$ ($I={\nat\times\nat}$)
 \end{itemize}

 }

\frame
  {   
    \frametitle{Indexing with structured objects}\label{Ch3:indexgraph}

 \begin{itemize}[<+->]
\item Even simpler than $MG$ is the graph $Ar$:
\begin{tikzcd}[ampersand replacement=\&]
L\arrow{r}{ar}\&R
\end{tikzcd}
\item $Ar$ is almost its own path category, just add identities
to get $\PP(Ar)$, commonly denoted as ${\cdot}{\to}{\cdot}$
\item The interpretation category $[Ar\to\CC]$ has:
   \begin{itemize}[<+->]
\item graph homomorphisms $f: Ar\to gr(\CC)$ as objects, being 
morphisms $f(ar): f(L) \to f(R)$ in $\CC$
\item natural transformations $\alpha: f\To g$, for any $f,g: Ar\to gr(\CC)$, as morphisms, 
being a pair $(\alpha_L,\alpha_R)$ such that
the naturality square holds for $ar: L\to R$, formally $f(ar);\alpha_R = \alpha_L;g(ar)$
   \end{itemize}
\item The interpretation category $[Ar\to\CC]$ and the functor category
$[{\cdot}{\to}{\cdot}\to\CC]$ are isomorphic
\item Uhh, what's that? Well, let's say small $\CC$, so in $\Cat$ (...)
 \end{itemize}

 }

\frame
  {   
    \frametitle{Arrow categories}\label{Ch3:arrowcats}

% \begin{itemize}[<+-> \item 

Given a category $\CC$, its \emph{arrow category} $\CC^\to$ has:
   \begin{itemize}[<+->]
\item triples $A = (A_L, ar^A, A_R)$ with $ar^A: A_L \to A_R$ in $\CC$ as objects
\item pairs $(f_L,f_R): A\to B$ with $f_L:A_L\to B_L$, $f_R:A_R\to B_R$ such that
$ar^A;f^R = f^L;ar^ B$, as morphisms (see left square under)
\item composition: if $(f_L,f_R): A\to B$ and $(g_L,g_R): B\to C$,
then $(f_L,f_R);(g_L,g_R) = (f_L;g_L,f_R;g_R): A\to C$, in a diagram:
\[
\begin{tikzcd}[ampersand replacement=\&]
%L \arrow[swap]{d}{ar} \& 
A_L  \arrow[swap]{d}{ar^A}\arrow{r}{f_L} \&\arrow{d}{ar^B} B_L
 \arrow{r}{g_L} \&\arrow{d}{ar^C} C_L\\
%R \&  
A_R \arrow{r}{f_R} \& B_R \arrow{r}{g_R} \& C_R
\end{tikzcd}
\]
\item  identities: $id^{\CC^\to}_{(A_L, ar^A, A_R)} = (id^{\CC}_{A_L},id^{\CC}_{A_R})$
(make a diagram)

\item ASS and ID are OK, inherited from $\CC$
   \end{itemize}


% \end{itemize}
}

\frame
  {   
    \frametitle{Modelling experiments}\label{Ch3:Modelling}

 \begin{itemize}[<+->]
\item E-graphs: script p.\ 103 ff
\item Path equations in a graph, in order to model, for example:
   \begin{itemize}[<+->]
\item 2-arrows (arrows between arrows)
\[M2G:~~~~~ 
\begin{tikzcd}[ampersand replacement=\&]
2Ar \arrow[yshift=0.5ex]{r}{dom} \arrow[swap, yshift=-0.5ex]{r}{cod} \& Arr \arrow[yshift=0.5ex]{r}{beg} \arrow[swap, yshift=-0.5ex]{r}{end} \& Nod,
\end{tikzcd}\]
with path equations in $\PP(M2G)$:
\[
[dom,beg]=[cod,beg] \text{~and~} [dom,end]=[cod,end]
\] 
\item Reflexive graphs
 \[MRG:~~~~~
\begin{tikzcd}[ampersand replacement=\&]
Arr \arrow[yshift=2ex]{r}{beg} \arrow[swap, yshift=-2ex]{r}{end}  \& \arrow{l}[description]{rfl} Nod,
\end{tikzcd}\] 
with path equation in $\PP(MRG)$: $[rfl,beg]=[rfl,end]=[]_{Nod}$ 
   \end{itemize}

 \end{itemize}

 }

\frame
  {   
    \frametitle{Path Equations}\label{Ch3:PathEqs}

 \begin{itemize}[<+->]
\item What do path equations like $[rfl,beg]=[rfl,end]=[]_{Nod}$ in $MRG$ mean?
\item For $F : \PP(MRG)\to\Set$ and paths $p_1,p_2$ in $MRG$, define $F\models p_1=p_2$
to mean $F^*(p_1) = F^*(p_2)$
\item Adding path equations to a (meta)graph means that we restrict to interpretations
that satisfy the path equations in the sense of $\models$
\item Example: $F\models [rfl,beg]=[rfl,end]=[]_{Nod}$, $F: MRG\to\Set$
means $F(rfl);F(beg)=F(rfl);F_E(end)=id_{F(Nod)}$, i.e., 
$F_E(beg)(F_E(rfl)(v)) = F_E(end)(F_E(rfl)(v)) = v$ for all $v\in F_V(Nod)$,
or as the commutative diagram on the right:
 \[
\begin{tikzcd}[ampersand replacement=\&]
MRG: ~ ~Arr \arrow[yshift=2ex]{r}{beg} \arrow[swap, yshift=-2ex]{r}{end}  \& \arrow{l}[description]{rfl} Nod,
\end{tikzcd}
~~ ~ ~ \begin{tikzcd}[ampersand replacement=\&]
F(Nod) \arrow{r}{F(rfl)}\arrow[swap]{d}{F(rfl)}\arrow{rd}{id}\& F(Arr) \arrow{d}{F(beg)} \\
F(Arr) \arrow{r}{F(end)} \& F(Nod)
\end{tikzcd}
\]
with path equation in $\PP(MRG)$: $[id,beg]=[id,end]=[]_{Nod}$ 

 \end{itemize}

 }

\subsection{3.3 Typing}


\frame
  {   
    \frametitle{Typing}\label{Ch3:Typing}

 \begin{itemize}[<+->]
\item More modelling: Petri nets (p. 111 of script) and 
Entity Relationship diagrams (p. 112 of script)
\item Both have different `types' of nodes and arrows: 
 \begin{itemize}
\item Petri-nets: $P$- and $T$-nodes and arrows $P\to T$ and $T\to P$
\item ER-diagrams: $E$-, $R$-, $A$-nodes and arrows $E\to A$ and $R\to A$ 
 \end{itemize}
\item This `typing' info can be encoded in a graph,
 \begin{itemize}
\item for Petri-nets the graph $PT$:
\begin{tikzcd}[ampersand replacement=\&]
P \arrow[yshift=0.5ex]{r}{in} \& 
T \arrow[yshift=-0.5ex]{l}{out}
\end{tikzcd}
\item for ER-diagrams the graph $RAE$:
\begin{tikzcd}[ampersand replacement=\&]
R \arrow{dr}{}\arrow{rr}{} \&\&  E \arrow{dl}{} \\
\& A
\end{tikzcd}
 \end{itemize}
\item Given a (type)graph $TG$,  a \emph{$TG$-typed graph} is a pair $(G,t)$ of a graph $G$ and a
homomorphism $t : G \to TG$ 
\item Examples: $TG{=}PT$ (Petri-nets), or $TG{=}RAE$ (ER-diagrams)
 \end{itemize}

 }

\frame
  {   
    \frametitle{Refinements of Typing}\label{Ch3:RefineTyp}

 \begin{itemize}[<+->]
\item In ER-diagrams, attributes can have attributes: add a loop around $A$
\item Example: `compound' attributes like first name, last name
\item Petri-nets are not `multi': at most one arrow between nodes
\item This can be encoded by requiring that $t: G\to PT$ in $(G,t)$
has the property that for all nodes $v,v'$ of $G$, $t : G \to PT$ is injective on the
arrows $v\to v'$ (works since $PT$ is not `multi')
\item For a (type)graph $TG$, let $(G,t_G)$ and  $(H,t_H)$ be $TG$-typed graphs.
Q: what would a morphism $\varphi: (G,t_G)\to(H,t_H)$ be?
\item A: a graph homomorphism $\varphi: G\to H$ \emph{preserving the typing}:
\begin{tikzcd}[ampersand replacement=\&]
G \arrow[swap]{rr}{\varphi}\arrow[swap]{dr}{t_G} \&\& H \arrow{dl}{t_H}\\
\& TG 
\end{tikzcd}
\item $TG$-typed graphs and morphisms form a category $\Gra/PT$
 \end{itemize}

 }

\frame
  {   
    \frametitle{Slice categories}\label{Ch3:SliceCat}

 \begin{itemize}[<+->]
\item In defining $\Gra/PT$ only categorical concepts are used
\item Slice categories are a generalization of  $\Gra/PT$
\item Given a category $\CC$ and an object $T$, the \emph{slice category}
$\CC/T$ has
   \begin{itemize}[<+->]
\item as objects all pairs $(A,t)$ with $A$ object of $\CC$ and $t: A\to T$
\item as morphisms $(A,T_A)\to(B,T_B)$ all morphisms $f: A\to B$ in $\CC$ such that
$t_A= f;t_B$ (see left triangle under)
\item identities and composition as in $\CC$, satisfying ID and ASS:

   \end{itemize}
\[
\begin{tikzcd}[ampersand replacement=\&]
A\arrow[swap]{d}{t_A}\arrow[loop left]{}{id_A}
\&\&A \arrow[swap]{r}{f}\arrow[swap]{dr}{t_A} \arrow[bend left]{rr}{f;g} \&
B \arrow[swap]{r}{g}\arrow[swap]{d}{t_B} \& 
C \arrow{dl}{t_C}  \\
T
\&\&\& T 
\end{tikzcd}
\]

\item Example: the category $I$-typed sets $\Gra/I$
\item Remark:
 \end{itemize}

 }


\frame
  {   
    \frametitle{Comments on script}\label{Ch3:comments}

 \begin{itemize}[<+->]
\item Example 3.2.10: "repaired"  means arrow $d$ reversed to keep $\gamma$
 \end{itemize}

 }



\end{document}

\myurl{en.wikipedia.org/wiki/Ordered_pair}
 \begin{itemize}
    \item emulation halts because of the partiality of $\delta$;
    \item emulation halts because of "head left" with head at first cell;
    \item emulation halts because of reaching a halting state;
    \item the emulated TM goes on forever.
 \end{itemize}
