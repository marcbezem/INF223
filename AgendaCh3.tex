\documentclass[handout]{beamer}
%\documentclass[slides]{beamer}
% Vary the color applet  (try out your own if you like)
%\colorlet{structure}{red!20!black}
%\beamertemplateshadingbackground{yellow!20}{white}
%\usepackage{beamerthemeshadow}
%\usepackage[utf8x]{inputenc} CONFLICT!
\usepackage{tikz}
%\usepackage[english,norsk,nynorsk]{babel}
\usepackage{tikz-cd}
\usetikzlibrary{trees}

\usepackage[all]{xy}
\usepackage{multicol}

%\setbeamertemplate{navigation symbols}{}++++++
%\setbeamertemplate{footline}[frame number]
\usetheme{Montpellier}


\input macros

\newcommand{\To}{\Rightarrow}
\newcommand{\Trt}{\stackrel{*}{\Rightarrow}}
\newcommand{\ToG}{\Rightarrow_G}
\newcommand{\redS}{{\color{red} S}}

\newcommand{\bfsf}[1]{{\boldsymbol{#1}}}
\newcommand{\Set}{\bfsf{Set}}
\newcommand{\Gra}{\bfsf{Graph}}
\newcommand{\CC}{\bfsf{C}}
\newcommand{\DD}{\bfsf{D}}
\newcommand{\EE}{\bfsf{E}}
\newcommand{\PP}{\bfsf{P}}
\newcommand{\HH}{\bfsf{H}}
\newcommand{\Nat}{\bfsf{Nat}}
\newcommand{\Incl}{\bfsf{Incl}}
\newcommand{\Rel}{\bfsf{Rel}}
\newcommand{\Mult}{\bfsf{Mult}}
\newcommand{\Mon}{\bfsf{Mon}}
\newcommand{\Cat}{\bfsf{Cat}}
\newcommand{\CAT}{\bfsf{CAT}}
\title[INF223 presentations]{}

\begin{document}

\section{Chapter 3}
\subsection{Natural transformations}
 
\frame
  {   
    \frametitle{Natural Transformations}\label{Ch3:NatTrans}

 \begin{itemize}[<+->]
\item Cold turkey: given functors $F,G: \CC\to\DD$, a \emph{natural transformation}
$\alpha: F\To G$ maps every object $A$ of $\CC$ to a morphism $\alpha_A : F(A)\to G(A)$
in $\DD$ such that, for every $f: A\to B$ in $\CC$ the following
\emph{naturality diagram} commutes:
\[
\begin{tikzcd}[ampersand replacement=\&]
A \arrow[swap]{d}{f} \& F(A)  \arrow[swap]{d}{F(f)}\arrow{r}{\alpha_A} \&\arrow{d}{G(f)} G(A)\\
B \& F(B)  \arrow{r}{\alpha_B} \& G(B)
\end{tikzcd}
\]
\item In $\HH$, Haskell-as-a-category, with types as objects and functions as morphisms:
   \begin{itemize}[<+->]
\item $[\_]_V$ maps any type $T$ to type $[T]$ of lists over $T$
\item $[\_]_E$ maps any $f: T\to T'$ to $map~f: [T] \to [T']$
\item $[\_]:\HH\to\HH$ is a functor, $reverse: [\_] \To [\_]$ is a natural transformation
   \end{itemize}
 \end{itemize}

 }

\frame
  {   
    \frametitle{What is natural about reversing lists?}\label{Ch3:reverse}

 \begin{itemize}[<+->]
\item Why is $reverse: [\_] \To [\_]$ a natural transformation?\\
Because this diagram commutes:
\[
\begin{tikzcd}[ampersand replacement=\&]
A \arrow[swap]{d}{f} \& {[A]}  \arrow[swap]{d}{map~f}\arrow{rr}{reverse_A} \&\& {[A]}\arrow{d}{map~f}\\
B \& {[B]}  \arrow{rr}{reverse_B}\& \& {[B]}
\end{tikzcd}
\]
\item What is natural about reversing lists?
That you can program this without knowing what the type of the elements is.
In other cultures, this is called \emph{generic programming},
or \emph{type polymorphism}
\item NB what naturality means depends on the categories $\CC,\DD$
 \end{itemize}

 }

\frame
  {   
    \frametitle{Yet another natural transformation}\label{Ch3:length}

 \begin{itemize}[<+->]
\item Functors can be constant
\item Example in $\HH$: the functor $C(\nat)$ 
   \begin{itemize}[<+->]
\item $C(\nat)_V$ maps any type $T$ to type $\nat$, $C(\nat)_V(T)$
\item $C(\nat)_E$ maps any $f: T\to T'$ to $id_\nat: \nat\to\nat$
   \end{itemize}
\item $[\_],C(\nat):\HH\to\HH$ are functors (why?), $length: [\_] \To C(\nat)$ 
is a natural transformation. Why? \vspace*{-0.5cm}
\item Because this diagram commutes:
\begin{tikzcd}[ampersand replacement=\&]
A \arrow[swap]{r}{f} \& B\\
{[A]}  \arrow[swap]{d}{length_A}\arrow{r}{map~f} \&{[B]}  \arrow{d}{length_B}\\
\nat \arrow{r}{id_\nat} \& \nat
\end{tikzcd}
\item What is natural about computing the length of a list?\\
That you can program $length_A$ uniformly in $A$.
 \end{itemize}

 }

\frame
  {   
    \frametitle{A hierarchy of (meta)models}\label{Ch3:metameta}

 \begin{itemize}[<+->]
\item A formal definition of the grammar class (metametamodel, e.g., CFGs
$(\mathcal T, \mathcal N, S, \mathcal R)$ with $S\in\mathcal N$ and
$\mathcal R \subseteq\mathcal N \times (\mathcal N \cup\mathcal T)^*$
\item A specific context-free grammar describing the syntax (metamodel)
\item A program written in a this specific syntax (model)
\item A machine performing a specific task (or not, that's reality :-)
 \end{itemize}

 }

\frame
  {   
    \frametitle{Metamodel of graphs}\label{Ch3:MG}

 \begin{itemize}[<+->]
\item Small graphs are given by two sets and two functions
\item More precisely: if the sets are $Nod,Arr$ we have $beg,end: Arr\to Nod$
\item This data has itself the form of a graph, a metamodel!
\item Define the metagraph $MG = 
\begin{tikzcd}[ampersand replacement=\&]
Arr \arrow[yshift=0.5ex]{r}{beg} \arrow[swap, yshift=-0.5ex]{r}{end} \& Nod,
\end{tikzcd} $ or:
   \begin{itemize}[<+->]
\item $MG_V = \set{Nod,Arr}$
\item $MG_E = \set{beg,end}$
\item $sc^{MG}(beg) = sc^{MG}(end) = Arr$
\item $tg^{MG}(beg) = tg^{MG}(end) = Nod$
   \end{itemize}
\item Small graphs are now graph homomorphisms $MG \to gr(\Set)$
\item Graph homomorphisms are now natural transformations (...)
 \end{itemize}

 }

\frame
  {   
    \frametitle{Exploring the idea of the metagraph MG}\label{Ch3:ideaMG}

 \begin{itemize}[<+->]
\item As graph \begin{tikzcd}[ampersand replacement=\&]
Arr \arrow[yshift=0.5ex]{r}{beg} \arrow[swap, yshift=-0.5ex]{r}{end} \& Nod,
\end{tikzcd} since there are no composable arrows,  $MG$ is almost its path category:
 just add identities  %($= []_{Arr}, []_{Nod}$)
\item Similarly, any graph homomorphism $MG\to gr(\CC)$ is almost a functor to
the category $\CC$: just map identities to identities
\item Generally: extend $\varphi: G\to gr(\CC)$ to functor $\varphi^*: \PP(G)\to\CC$
\item A graph homomorphism $F : MG\to gr(\Set)$ consists of
   \begin{itemize}[<+->]
\item a set $F_V(Nod)$ the we call the set of nodes
\item a set $F_V(Arr)$ the we call the set of arrows
\item a map $F_E(beg) : F_V(Arr)\to F_V(Nod)$, the source map $sc$
\item a map $F_E(end) : F_V(Arr)\to F_V(Nod)$, the target map $tg$
   \end{itemize}
and in this way you can get any small graph by a suitable $F$!
\item We call such $F$'s \emph{interpretations} of $MG$ % , also written $F : MG\to \Set$
 \end{itemize}

 }

\frame
  {   
    \frametitle{Interpretations  of a graph}\label{Ch3:interpretations}

 \begin{itemize}[<+->]
\item An \emph{interpretation} of a graph $M$ in a category $\CC$
is a graph homomorphism $F: M\to gr(\CC)$ also written $F : M\to \CC$
\item Example: small hypergraphs, with arrows  between sets of nodes
are interpretations $M\to \Mult$, 
a graph homomorphism $F : MG\to gr(\Set)$ consists of
   \begin{itemize}[<+->]
\item a set $F_V(Nod)$ the we call the set of nodes
\item a set $F_V(Arr)$ the we call the set of arrows
\item a multimap $F_E(beg) : F_V(Arr)\to F_V(Nod)$, the source map:  
$F_E(beg)(a)\in {\color{red}\powset} F_V(Nod)$ for all $a\in F_V(Arr)$
\item a multimap $F_E(end) : F_V(Arr)\to F_V(Nod)$, the target map: 
 $F_E(end)(a)\in {\color{red}\powset} F_V(Nod)$  for all $a\in F_V(Arr)$,
   \end{itemize}
\item In this way you can get any small {\color{red}hyper}graph by a suitable $F$!
\item By varying the category $\CC$ we get lots of interpretations
 \end{itemize}

 }

\frame
  {   
    \frametitle{Homomorphisms recovered (1)}\label{Ch3:Homs1}

 \begin{itemize}[<+->]
\item With the {interpretations} of a graph $M$ in a the category $\Set$
being small graphs, where are the graph homomorphisms?
\item Let $F,G : M\to \CC$ be two interpretations. They can be promoted
to functors $F^*,G^* : \PP(M)\to \CC$. Any natural transformation
$\alpha: F^* \To G^*$ satisfies
\[
\begin{tikzcd}[ampersand replacement=\&]
v \arrow[swap]{d}{[f_1,...,f_n]} \&\& {F^*(v)}  \arrow[swap]{d}{F^*([f_1,...,f_n])}\arrow{rr}{\alpha_v} 
\&\& {G^*(v})\arrow{d}{G^*([f_1,...,f_n])}\\
u \&\&{F^*(u)}  \arrow{rr}{\alpha_u}\& \& {G^*(u)}
\end{tikzcd}
\]
where $F^*([f_1,...,f_n]) = F(f_1);...;F(f_n)$ and likewise for $G$
\item This will be simplified from $\PP(G)$ to $G$ in the next slide
 \end{itemize}

 }

\frame
  {   
    \frametitle{Homomorphisms recovered (2)}\label{Ch3:Homs2}

 \begin{itemize}[<+->]
\item Recall we have {interpretations} $F,G : M\to \CC$ and a natural transformation
$\alpha: F^* \To G^*$. By list recursion and using $F^*(v) = F(v)$ the following
naturality diagram is equivalent:
\[
\begin{tikzcd}[ampersand replacement=\&]
v \arrow[swap]{d}{f} \&\& {F(v)}  \arrow[swap]{d}{F(f)}\arrow{rr}{\alpha_v} 
\&\& {G(v})\arrow{d}{G(f)}\\
u \&\&{F(u)}  \arrow{rr}{\alpha_u}\& \& {G(u)}
\end{tikzcd}
\]
\item This is about data in $F,G : M\to \CC$ only and the script calls 
such $\alpha$ a natural transformation of interpretations
\item The next slide recovers graph homomorphisms as
the special case $M=MG$ and $\CC=\Set$, but the generality will be useful
 \end{itemize}

 }

\frame
  {   
    \frametitle{Homomorphisms recovered (3)}\label{Ch3:Homs3}

 \begin{itemize}[<+->]
\item $MG = 
\begin{tikzcd}[ampersand replacement=\&]
Arr \arrow[yshift=0.5ex]{r}{beg} \arrow[swap, yshift=-0.5ex]{r}{end} \& Nod,
\end{tikzcd} $ $F,G: MG\to\Set$, $\alpha:F\To G$
\[
\begin{tikzcd}[ampersand replacement=\&]
Arr \arrow[swap]{d}{beg} \&\& {F(Arr)}  \arrow[swap]{d}{F(beg)}\arrow{rr}{\alpha_{Arr}} 
\&\& {G(Arr})\arrow{d}{G(beg)}\\
Nod \&\&{F(Nod)}  \arrow{rr}{\alpha_{Nod}}\& \& {G(Nod)}
\end{tikzcd}
\]
+
\[
\begin{tikzcd}[ampersand replacement=\&]
Arr \arrow[swap]{d}{end} \&\& {F(Arr)}  \arrow[swap]{d}{F(end)}\arrow{rr}{\alpha_{Arr}} 
\&\& {G(Arr})\arrow{d}{G(end)}\\
Nod \&\&{F(Nod)}  \arrow{rr}{\alpha_{Nod}}\& \& {G(Nod)}
\end{tikzcd}
\]
= graph homomorphism represented by $\alpha$ 
 \end{itemize}

 }

\frame
  {   
    \frametitle{An unnatural transformation}\label{Ch3:unnat}

 \begin{itemize}[<+->]
\item Is removing duplicates $nodup_A : [A] \to [A]$ natural in $A$?
\item No, this diagram need not commute:
\[
\begin{tikzcd}[ampersand replacement=\&]
A \arrow[swap]{d}{f} \& {[A]}  \arrow[swap]{d}{map~f}\arrow{rr}{nodup_A} \&\& {[A]}\arrow{d}{map~f}\\
B \& {[B]}  \arrow{rr}{nodup_B}\& \& {[B]}
\end{tikzcd}
\]
\item Why not?
\item Fix: codomain is the wrong category, take $[\_],\powset : \Set\to\Set$ as functors
and it is easy to define a natural $list2set : [\_]\To\powset$,
$\left\{\begin{array}{ll}list2set_A([]) &= \emptyset\\list2set_A(a:as) &=\set{a}\cup list2set_A(as)
\end{array}\right.$
 \end{itemize}

 }

\frame
  {   
    \frametitle{Sorting, naturally}\label{Ch3:sorting}

 \begin{itemize}[<+->]
\item For sorting we need to equip the type $A$ with an ordering $\leq$
\item This changes the category to $\Set^{<}$: objects are ordered sets (or types) and
morphisms are order-preserving maps, i.e., $f: (A,\leq)\to (B,\preceq)$ 
such that $f(a)\preceq f(a')$ for all $a\leq a'$. 
\item Then we should also order $[A]$! Take $\leq_{lex}$, and observe:
if $l \leq_{lex} l'$, then $map~f~l \preceq_{lex} map~f~l'$ for any $f$ as above.
\item This shows that we have a functor $[\_]^{<} : \Set^{<} \to \Set^{<}$
   \begin{itemize}[<+->]
\item mapping object $(A,\leq)$ to object $([A],\leq_{lex})$, 
\item just $map$ on the morphisms of $\Set^{<}$ (order-preserving $f$'s).
   \end{itemize}
\item Sorting before or after mapping $f$ makes no difference:
\[
\begin{tikzcd}[ampersand replacement=\&]
(A,\leq) \arrow[swap]{d}{f} \& {[A]}  \arrow[swap]{d}{map~f}\arrow{rr}{sort_A} \&\& {[A]}\arrow{d}{map~f}\\
(B,\preceq) \& {[B]}  \arrow{rr}{sort_B}\& \& {[B]}
\end{tikzcd}
\]
 \end{itemize}

 }

\frame
  {   
    \frametitle{An unnatural transformation}\label{Ch3:unnat}

 \begin{itemize}[<+->]
\item Is removing duplicates $nodup_A : [A] \to [A]$ natural in $A$?
\item No, this diagram need not commute:
\[
\begin{tikzcd}[ampersand replacement=\&]
A \arrow[swap]{d}{f} \& {[A]}  \arrow[swap]{d}{map~f}\arrow{rr}{nodup_A} \&\& {[A]}\arrow{d}{map~f}\\
B \& {[B]}  \arrow{rr}{nodup_B}\& \& {[B]}
\end{tikzcd}
\]
\item Why not?
\item Fix: codomain is the wrong category, take $[\_],\powset : \Set\to\Set$ as functors
and it is easy to define a natural $list2set : [\_]\To\powset$,
$\left\{\begin{array}{ll}list2set_A([]) &= \emptyset\\list2set_A(a:as) &=\set{a}\cup list2set_A(as)
\end{array}\right.$
 \end{itemize}

 }




\frame
  {   
    \frametitle{Comments on script}\label{Ch3:comments}

 \begin{itemize}[<+->]
\item 
 \end{itemize}

 }



\end{document}

\myurl{en.wikipedia.org/wiki/Ordered_pair}
 \begin{itemize}
    \item emulation halts because of the partiality of $\delta$;
    \item emulation halts because of "head left" with head at first cell;
    \item emulation halts because of reaching a halting state;
    \item the emulated TM goes on forever.
 \end{itemize}
