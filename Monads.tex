\documentclass[handout]{beamer}
%\documentclass[slides]{beamer}
% Vary the color applet  (try out your own if you like)
%\colorlet{structure}{red!20!black}
%\beamertemplateshadingbackground{yellow!20}{white}
%\usepackage{beamerthemeshadow}
%\usepackage[utf8x]{inputenc} CONFLICT!
\usepackage{tikz}
%\usepackage[english,norsk,nynorsk]{babel}
\usepackage{tikz-cd}
\usetikzlibrary{trees}

\usepackage[all]{xy}
\usepackage{multicol}

%\setbeamertemplate{navigation symbols}{}++++++
%\setbeamertemplate{footline}[frame number]
\usetheme{Montpellier}


\input macros

\newcommand{\To}{\Rightarrow}
\newcommand{\Trt}{\stackrel{*}{\Rightarrow}}
\newcommand{\ToG}{\Rightarrow_G}
\newcommand{\redS}{{\color{red} S}}

\newcommand{\bfsf}[1]{{\boldsymbol{#1}}}
\newcommand{\Set}{\bfsf{Set}}
\newcommand{\Gra}{\bfsf{Graph}}
\newcommand{\CC}{\bfsf{C}}
\newcommand{\DD}{\bfsf{D}}
\newcommand{\EE}{\bfsf{E}}
\newcommand{\PP}{\bfsf{P}}
\newcommand{\HH}{\bfsf{H}}
\newcommand{\Nat}{\bfsf{Nat}}
\newcommand{\Incl}{\bfsf{Incl}}
\newcommand{\Rel}{\bfsf{Rel}}
\newcommand{\Mult}{\bfsf{Mult}}
\newcommand{\Mon}{\bfsf{Mon}}
\newcommand{\Cat}{\bfsf{Cat}}
\newcommand{\CAT}{\bfsf{CAT}}

\newcommand{\Kp}[1]{{\langle #1 \rangle}}
\newcommand{\Kc}{;\!;}


\title[INF223 presentations]{}

\begin{document}

\section{Monads}
\subsection{Introduction}
 
\frame
  {   
    \frametitle{Monads}\label{Mon5:Intro}

 \begin{itemize}[<+->]
\item Monad is a notion in CT that turned out to be useful in programming
\item Monad was defined in CT in the 1950's and 1960's
\item A breakthrough paper for application in CS was 
\href{https://person.dibris.unige.it/moggi-eugenio/ftp/ic91.pdf}{\color{blue}E. Moggi, Notions of computation and monads, I\&C, 93(1), 1991}
\item We start by giving an example, and then carefully abstract from that
to arrive at a categorical notion
\item Useful links are:
 \begin{itemize}
    \item \myurl{https://ncatlab.org/nlab/show/monad+\%28in+computer+science\%29}
    \item \myurl{https://ncatlab.org/nlab/show/extension+system}
 \end{itemize}
 \end{itemize}

 }

\frame
  {   
    \frametitle{Example of a monad (one)}\label{Mon5:ExaSetMone}

 \begin{itemize}[<+->]
\item We work in the category $\Set$ and fix a monoid $(M,1,*)$
\item The example models functions that have side-effects, e.g., writing a
string (strings with the empty string and concatenation form a monoid)
\item Operation $T$ on sets is defined by $T(X)= X\times M$
\item Map $f : X\to Y$ with side-effect $p : X\to M$ is modelled as $\Kp{f,p} : X\to T(Y)$
%$p$ gives the monoid value as a side-effect of $f$
\item For every set $X$ we define $unit_X: X\to T(X)$ mapping $x\in X$ to $(x,1)$
\item For the moment $unit_X$ is just an example, but it will turn out to
be some kind of unit element
 \end{itemize}
 }

\frame
  {   
    \frametitle{Example of a monad (two)}\label{Mon5:ExaSetMtwo}

 \begin{itemize}[<+->]
\item For $\Kp{f,p}: X\to T(Y)$ and $\Kp{g,q}: Y\to T(Z)$ we define a special
composition $\Kp{f,p}\Kc \Kp{g,q} = \Kp{f;g,r} : X\to T(Z)$ with $$r(x)=p(x)*q(f(x))$$
This expresses that the monoid-valued-side-effect is cumulative,
we multiply the old value $p(x)$ by the new value $q(f(x))$,
or, we append the string $q(f(x))$ to the string $p(x)$.
\item Indeed we get a unit law: for $\Kp{f,p}: X\to T(Y)$ we have
$$unit_X\Kc \Kp{f,p}  = \Kp{f,p} = \Kp{f,p}\Kc unit_Y$$
\item Also, the special composition $\Kc$ is associative
\item Even though we have left and right units, $\Kc$ is not a multiplica- tion.
We don't have a new \emph{monoid}, but we do have a \emph{monad}.
 \end{itemize}

 }

\frame
  {   
    \frametitle{Example of a monad (three)}\label{Mon5:ExaSetMthree}

 \begin{itemize}[<+->]
\item Preparing for generalization, we abstract from the example
\item For $\Kp{f,p} : X\to T(Y)$ we define $\Kp{f,p}^T : T(X)\to T(Y)$ by
$$\Kp{f,p}^T(x,v) = (f(x),v*p(x))$$
\item Now we can extend $T$ to a functor by setting $$T(f) = (f; unit_Y)^T : T(X)\to T(Y)$$
\item We verify:
 \begin{itemize}
    \item $T(id_X) = (id_X; unit_X)^T = (unit_X)^T = id_{T(X)}$
    \item for all $f:X\to Y$ and $g:Y\to Z$, $T(f;g) = (f;g; unit_Z)^T$ 
and $T(f); T(g) = (f; unit_Y)^T ; (f; unit_Z)^T$
 \end{itemize}

 \end{itemize}

 }



\end{document}

\myurl{en.wikipedia.org/wiki/Ordered_pair}
 \begin{itemize}
    \item emulation halts because of the partiality of $\delta$;
    \item emulation halts because of "head left" with head at first cell;
    \item emulation halts because of reaching a halting state;
    \item the emulated TM goes on forever.
 \end{itemize}
