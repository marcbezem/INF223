\documentclass[handout]{beamer}
%\documentclass[slides]{beamer}
% Vary the color applet  (try out your own if you like)
%\colorlet{structure}{red!20!black}
%\beamertemplateshadingbackground{yellow!20}{white}
%\usepackage{beamerthemeshadow}
%\usepackage[utf8x]{inputenc} CONFLICT!
%\usepackage[english,norsk,nynorsk]{babel}
\usepackage{tikz}
\usetikzlibrary{trees}

\usepackage[all]{xy}
\usepackage{multicol}

%\setbeamertemplate{navigation symbols}{}++++++
%\setbeamertemplate{footline}[frame number]
\usetheme{Montpellier}
\setbeamertemplate{footline}[frame number]


\input macros

\newcommand{\To}{\Rightarrow}
\newcommand{\Trt}{\stackrel{*}{\Rightarrow}}
\newcommand{\ToG}{\Rightarrow_G}
\newcommand{\redS}{{\color{red} S}}

\newcommand{\bfsf}[1]{{\boldsymbol{#1}}}
\newcommand{\Set}{\bfsf{Set}}
\newcommand{\Gra}{\bfsf{Graph}}
\newcommand{\CC}{\bfsf{C}}
\newcommand{\DD}{\bfsf{D}}
\newcommand{\Nat}{\bfsf{Nat}}
\newcommand{\Incl}{\bfsf{Incl}}
\newcommand{\Rel}{\bfsf{Rel}}


\title[INF223 presentations]{}

\begin{document}

\section{Chapter 2}
\subsection{Directed Multigraphs}
 
\frame
  {   
    \frametitle{Directed Multigraphs}\label{Ch2:DMG}

 \begin{itemize}[<+->]
\item A (directed multi)graph $G$ consists of the following data:
   \begin{itemize}[<+->]
\item A collection $G_V$ of vertices (also called nodes)
\item A collection $G_E$ of (directed) edges (also called arrows)
\item A function $sc^G : G_ E \to G_V$ called the source function
\item A function $tg^G : G_ E \to G_V$ called the target function
   \end{itemize}
\item Any $f:G_E$ is an arrow from node $sc^G(f)$ to node $tg^G(f)$
\item We also write $f:v\to u$ and $v \stackrel{f}{\to} u$ to express
that $sc^G(f)=v$ and $tg^G(f)=u$, colloquially  "from $v$ to $u$"
\item Graphs are abundant in CS (un/directed, multi-, labelled, ...)
\item Simple graphs: at most one edge $f:v\to u$, for any $v,u\in G_ V$
\item For simple graphs we can put $G_ E \subseteq G_V\times G_V$
 \end{itemize}

 }

\frame
  {   
    \frametitle{Examples of  directed multigraphs}\label{Ch2:DMGexamples}

 \begin{itemize}[<+->]
\item Several examples from the script on Whiteboard
\item As we will see, categories are directed multigraphs + extras
\item With/out decoration: class diagrams, flow charts, state transition systems,
entity relationship diagrams, ...
\item Successor relation on $\nat$: $0\to 1,~1\to 2,~2\to 3,\ldots$
\item Any binary relation on a set forms a simple graph
\item (span) $A\stackrel{\pi_1}{\leftarrow} A\times B \stackrel{\pi_2}{\to} B$,
(cospan) $A\stackrel{\kappa_1}{\to} A\uplus B \stackrel{\kappa_2}{\leftarrow} B$
\item A (large) multigraph: nodes are sets, arrrows are functions $f:A\to B$,
with $A,B$ sets
 \end{itemize}

 }

\frame
  {   
    \frametitle{Graph Operations}\label{Ch1:GraphOps}

 \begin{itemize}[<+->]
\item Nice intro in script: working on a graph as an abstract software model
\item At some moment in time we have graph $G$, what next?
   \begin{enumerate}[<+->]
\item Add nodes and/or arrows and get a `supergraph' $H$ (provided all new edges
have source and target)
\item Delete nodes and/or arrows and get a `subgraph'
(delete all arrows `orphaned' by deleting nodes)
\item Rename nodes and/or arrows, and get an `isomorphic' graph
\item Collapse (make equal) nodes and/or arrows, taking care we get a graph
($f=g$ only if $f,g: u\to v$)
   \end{enumerate}
\item Examples on Whiteboard, illustrating the many different cases
 \end{itemize}

 }

\frame
  {   
    \frametitle{Graph Homomorphisms}\label{Ch1:GraphHoms}

 \begin{itemize}[<+->]
\item The operations on the previous slide lead to two new notions:
 \begin{itemize}[<+->]
\item $G\sqsubseteq H$ (sub/super graph) if $G_V\subseteq H_V$, $G_E\subseteq H_E$,
and for all $f: v\to u$ in $G$ we have $f: v\to u$ in $H$
\item $\varphi: G\to H$ graph homomorphisme if $\varphi$ maps nodes to nodes and arrows
to arrows such that for all $f: v\to u$ in $G$ we have $\varphi(f): \varphi(v)\to \varphi(u)$ in $H$
 \end{itemize}
\item Example: the identity graph homomorphism $id_G : G\to G$
\item Actually, for all graphs $G,H$ with $G_V \subseteq H_V$ and $G_E \subseteq H_E$, 
$G$ is a subgraph of $H$
iff the two inclusion maps together form a homomorphism
\item $\varphi: G\to H$ is a graph isomorphism if $\varphi$ is a graph
homomor- phism and there is  a graph homomorphism $\psi: H\to G$ such that
$\varphi\circ\psi$ and $\psi\circ\varphi$ are identity maps (cf.\ renaming above) 
 \end{itemize}

 }


\frame
  {   
    \frametitle{Opposite graph}\label{Ch1:oppgraph}

 \begin{itemize}[<+->]
\item Given a graph $G$, the opposite graph $G^{opp}$ has the same nodes
and arrows, but the directions of all arrows is reversed
\item This boils down to interchanging the $sc$ and $tg$ functions
%\item Examples: the opposite of a span is a cospan;  the opposite of a cospan is a span
\item Clearly: ${G^{opp}}^{opp} = G$
\item NB1: $(b \stackrel{c}{\leftarrow} a) = (a \stackrel{c}{\to} b) \neq  
(b \stackrel{c}{\to} a) = (a \stackrel{c}{\leftarrow} b) $
if $a\neq b$
\item NB2: As graphs with two nodes and one arrow, ${a \stackrel{c}{\to} b}$
and ${b \stackrel{c}{\to} a}$ are each other's opposite, and happen to be isomorphic
\item NB3: opposite graphs need not be isomorphic in general, e.g., the span
${b \stackrel{c}{\leftarrow} a}\stackrel{d}{\to}a$ is not isomorphic to its cospan if $a\neq b$
\item Elaboration on Whiteboard
 \end{itemize}

 }

\section{Chapter 2}
\subsection{Categories}


\frame
  {   
    \frametitle{Composition and identities in set theory}\label{Ch2:comp-id}

 \begin{itemize}[<+->]
\item For all maps $f:A\to B$ and $f':A\to B$ we have $f=f'$ if $f(a)=f'(a)$ for all $a\in A$
\item Two notations of composition of maps $f:A\to B$ and $g: B\to C$,
both denoting the same map $A\to C$:
   \begin{itemize}[<+->]
\item Applicative order: ${g\circ f} : {A\to C}$ reflecting $a\mapsto g(f(a))$
\item Diagrammatric order: ${f;g} : A\to C$ reflecting $A\stackrel{f}{\to}B\stackrel{g}{\to}C$
   \end{itemize}
\item The original sin here is the convention to write the function before its argument
instead of after: $f(a)$ instead of $af$,
while at the same time reading from left to right
\item Anyway: $f;(g;h) = (f;g);h$, and so $h\circ(g\circ f) = (h\circ g)\circ f$,
for all $A\stackrel{f}{\to}B\stackrel{g}{\to}C\stackrel{h}{\to}D$
\item Also: $id_A ; f = f$ and $f; id_B = f$ for all $f:A\to B$
 \end{itemize}

 }

\frame
  {   
    \frametitle{Composition of graph homomorphisms}\label{Ch2:Ghom-comp}

 \begin{itemize}[<+->]
\item A graph homomorphism $\varphi: G\to H$ is a pair of maps
$\varphi_V : G_V \to H_V$ and $\varphi_E : G_E \to H_E$ preserving graph structure 
\item Similarly, a graph homomorphism $\psi: H\to K$ is a pair of maps
$\psi_V : H_V \to K_V$ and $\psi_E : H_E \to K_E$ preserving graph structure
\item Define $\varphi;\psi$ to be the pair of $\varphi_V;\psi_V$ and $\varphi_E;\psi_E$
\item Then: $\varphi;\psi$ is a graph homomorphism $G\to K$
\item Proof is spelled out in detail in the script
\item A lighter touch: if $f: v\to u$ in $G$, 
then $\varphi(f): \varphi(v)\to \varphi(u)$ in $H$
since $\varphi$ is a graph homomorphism $G\to H$, 
hence we have $\psi(\varphi(f)): \psi(\varphi(v))\to \psi(\varphi(u))$ in $K$
since $\psi$ is a graph homomorphism $H\to K$, 
which is what we need 
%$\varphi;\psi(f)): \varphi;\psi(v))\to \varphi;\psi(u))$ in $K$
\item Also: $id_G ; \varphi = \varphi$ and $\varphi; id_H = \varphi$ for all $\varphi:G\to H$
 \item Also: associativity $\varphi;(\psi;\rho) = (\varphi;\psi);\rho$
 \end{itemize}

 }


\frame
  {   
    \frametitle{Categories}\label{Ch2:categories}

 \begin{itemize}[<+->]
\item A category $\CC = (\CC_V,\CC_E,sc^\CC, tg^\CC, id^\CC,(\_;^\CC\_))$ consists of the following data:
   \begin{enumerate}[<+->]
\item A graph $gr(\CC) =  (\CC_V,\CC_E,sc^\CC, tg^\CC)$ where nodes are now also called objects
and arrows also morphisms
\item A (partial) composition operation $(\_;\_)$ mapping $f: A\to B$, $g: B\to C$
to $(f;g): A\to C$, for all objects $A,B,C$
\item A map $id^\CC$ mapping any object $A$ to a morphism $id^\CC_A: A\to A$
   \end{enumerate}
satisfying the identity laws and the associativity law:
   \begin{itemize}[<+->]
\item(ID) if $f: A\to B$, then $id^\CC_A;f = f$ and  $f;id^\CC_B = f$
\item(ASS) if $A \stackrel{f}{\to} B \stackrel{g}{\to} C \stackrel{h}{\to}D $, then $f;(g;h)=(f;g);h$
   \end{itemize}
\item NB $\CC \neq C$, we may omit superscripts $\CC$,
we write $\CC(A,B)$ for the collection of morphisms $A\to B$
\item Indentities are uniquely determined by the identity laws
 \end{itemize}

 }

\frame
  {   
    \frametitle{Examples of Categories}\label{Ch2:ExaCat}

 \begin{itemize}[<+->]
\item Haskell (idealized), with types as objects and functions as morphisms, 
the polymorphic identity function and polymorphic \myurl{https://wiki.haskell.org/Function_composition}
\item $\Set$ with sets as objects, functions as morphisms,
$id_A: A\to A$ the identity function, and ``$;$'' as on slide \ref{Ch2:comp-id}
\item $\Gra$ with graphs as objects, graph homomorphisms as morphisms,
$id_G: G\to G$ the identity graph homomorphism on slide \ref{Ch1:GraphHoms}, 
and composition ``$;$' on slide \ref{Ch2:Ghom-comp}
\item Cartesian product, coproduct and epi-mono factorization are additional
features that a category may/not have. For example, $\Set$ with only injections,
with only surjections, or with only bijections are all three categories, but
with different features (see next slide)
 \end{itemize}

 }

\frame
  {   
    \frametitle{More Examples of Categories}\label{Ch2:MoreExaCat}

 \begin{itemize}[<+->]
\item Sets with injections ($A\times B \stackrel{\pi_1}{\to} A$ is not always injective)
\item Sets with surjections ($ A\stackrel{\kappa_1}{\to} A\uplus B$ is not always surjective)
\item Sets with bijections (both problems)
\item One set $A$ with all maps $A\to A$ forms a category $\mathsf{A}$
which does not satisfy epi-mono factorization if $A$ contains more than one element
\item A category is called small if the collection of objects is a set, and locally small
if $\CC(A,B)$ is a set for all objects $A,B$
\item Examples: $\mathsf{A}$ is both small and locally small, $\Set$ is only locally small,
but not small, since there is no set of all sets
 \end{itemize}
 }

\frame
  {   
    \frametitle{Isomorphisms}\label{Ch2:Isos}

 \begin{itemize}[<+->]
\item A morphism $f : A\to B$ in a category $\CC$ is called an iso (isomorphism)
if there exists a morphism $g : B\to A$ such that $f;g = id^\CC_A$ and
$g;f = id^\CC_B$
\item If it exists, the morphism $g$ above is called the inverse of $f$
\item If $g$ is the inverse of $f$, then $g$ is unique: if $g'$ is also
an inverse of $f$, then $g = id_B;g = (g';f);g = g';(f;g) = g';id_A = g'$.
\item Examples: in any category, $id_A$ is an iso and $id_A^{-1} = id_A$;
in $\Set$, iso = bijection; in $\Gra$, iso = graph isomorphism
\item If there is an iso $f: A\to B$, we call $A$ and $B$ isomorphic, 
denoted $A\simeq B$
\item If  $A \stackrel{f}{\to} B \stackrel{g}{\to} C$ and $f,g$ are isos, then
$f;g$ is iso and $(f;g)^{-1} = g^{-1}; f^{-1}$
(Proof on Whiteboard, add loop $id_B$)
 \end{itemize}

 }

\frame
  {   
    \frametitle{Even More Examples of Categories}\label{Ch2:evenMoreExaCat}

 \begin{itemize}[<+->]
\item The empty category: no objects, and hence no morphisms
\item Discrete categories: only identity morphisms, $\bfsf{n}$ denotes
the discrete category with $n$ objects (later: names don't matter)
\item Finite categories and pictorial diagrams: WYSIWYG?
\item Complete underlying graph: WYSCanBeWrong, WYSIAllYG (objects and morphisms,
extra info may be needed for ;)
\item Several examples on Whiteboard
\item Really nice: 2.2.5, exercise 5
 \end{itemize}

 }

\section{Chapter 2}
\subsection{Sets and Relations}

\frame
  {
    
    \frametitle{Binary Relation Therapy (after MNF130)}\label{Ch2:BinRels}

 \begin{itemize}[<+->]\label{right-left-invariance}
    \item Binary relation $R$ is an \emph{equivalence} on a set $S$ if:
    \begin{itemize}[<+->]
    \item $R$ is \emph{reflexive}: $\forall x\in S~(xRx)$, and
    \item $R$ is \emph{symmetric}: $\forall x,y\in S~(xRy \implies yRx)$, and
    \item $R$ is \emph{transitive}: $\forall x,y,z\in S~(xRy \wedge yRz \implies xRz)$.
    \end{itemize}
    \item Any equivalence relation $R$ on $S$ \emph{partitions} $S$ 
    in \emph{classes} $[x]_R = \set{y \mid xRy}$; 
    the \emph{quotient} is $S/R = \set{[x]_R \mid x\in S}$
    \item Binary relation $R$ on $S$ is
    \begin{itemize}[<+->]
     \item a \emph{preorder} if $R$ is reflexive and transitive
     \item a \emph{partial order} if $R$ is a preorder and antisymmetrisk:
     $\forall x,y\in S~(xRy \wedge yRx \implies x=y)$
     \item a \emph{(total) order} if $R$ is a partial order and total:
     $\forall x,y\in S~(xRy \vee yRx)$
    \end{itemize}
 \end{itemize}

}

\frame
  {
    
    \frametitle{Binary Relation Therapy (ctnd)}\label{Ch2:BinRelsCtnd}

 \begin{itemize}[<+->]\label{right-left-invariance}
    \item Examples
    \begin{itemize}[<+->]
    \item total order: $\leq$ on a set of numbers
    \item partial order: $\subseteq$ on sets % often on the set $\powset S$
    \item preorder: $\sqsubseteq$ on a set persons defined by age comparison
%$p\sqsubseteq p'$  if $p'$ is older than $p$
    \end{itemize}
    \item Any equivalence relation $R$ on $S$ \emph{partitions} $S$ 
    in \emph{classes} $[x]_R = \set{y \mid xRy}$; 
    the \emph{quotient} is $S/R = \set{[x]_R \mid x\in S}$
    \item Example of the an equivalence relation: let $R$ be a preorder
and define $p\simeq p'$ if $(p R p' \wedge p' R p)$. Then $\simeq$ is an equivalence
relation.In the case of $R$ being ${\sqsubseteq}$ above, we have that  
$[p]_\simeq = \set{p' \mid p\simeq p'}$  is the age class of $p$. Moreover,
we can define a relation $[R]$ on classes by $[p]\,[R]\, [q] $ if $pRq$.
The relation $[R]$ is a partial ordering of the classes, 
explaining why $R$ is called preorder. Age classes are even totally ordered.
 \end{itemize}

}

\frame
  {   
    \frametitle{Subcategories}\label{Ch2:Subcategories}

 \begin{itemize}[<+->]
\item A saying: in CT all things that ought be true are in fact true
\item Since a category is a graph + extras (id, ``;'', ID, ASS), a subcategory is ...
indeed a subgraph with the extras coming from the category
\item A category $\DD$ is  a subcategory of a category $\CC$ if 
   \begin{enumerate}[<+->]
\item $gr(\DD) \sqsubseteq gr(\CC)$, i.e., $gr(\DD)$ is a subgraph of $gr(\CC)$
\item $(f;^\DD g) =(f;^\CC g)$ for all  objects $A,B,C$ of $\DD$ and $f\in\DD(A,B)$, $g\in\DD(B,C)$
\item $id^\DD_A =  id^\CC_A$ for all objects $A$ of $\DD$
   \end{enumerate}
\item An equivalent definition of subcategory is obtained by requiring that
$\DD_V$ and $\DD_E$ are closed under $sc^\CC,tg^\CC, ;^\CC, id^\CC$
\item Examples on next slide
 \end{itemize}

 }

\frame
  {   
    \frametitle{Ever More Examples of Categories}\label{Ch2:everMoreExaCat}
 \begin{itemize}[<+->]
\item The category $\Incl$: maps $in_{A,B} :A\to B$ if $A\subseteq B$ ($A,B$ sets)
\item The category $\Nat$: one morphism $n\to m$ if $n\leq m$  ($n,m\in\nat$)
\item The category $\bfsf{Pow}(S)$: small subcategory of $\Incl$ with object
set $\powset{S}$, given some fixed but arbitrary set $S$
\item More general: preorder category and partial order category
\item The category $\Rel$: morphisms $R\subseteq A\times B$ ($A,B$ sets),
composition of relations (DBs: join!), $id_A = \set{(a,a)\mid a\in A}$
\item Subcategory of $\Rel$: morphisms $A\to B$ are partial functions
(a partial function may leave $f(a)$ undefined)
 \end{itemize}

 }


\frame
  {   
    \frametitle{More Basics on Sets}\label{Ch2:SetBasics}


 \begin{itemize}[<+->]\label{right-left-invariance}
    \item Binary relation $R\subseteq A\times B$ from $A$ to $B$ is
    \begin{itemize}[<+->]
    \item \emph{single-valued} if  $\forall a\in A,~b,b'\in B~(aRb \wedge aRb' \implies b=b')$
    \item \emph{total} if  $\forall a\in A~\exists b\in B~(aRb)$
    \item a \emph{partial function} if  $R$ is single-valued
    \item a \emph{(total) function} if  $R$ is single-valued and total
    \end{itemize}
   \item If $R\subseteq A\times B$ and $S\subseteq B\times C$ then their \emph{composition}          
     $R;S$ is \[\set{(a,c) \mid \exists b\in B~(aRb\wedge bRc)}\subseteq A\times C\]
    \item The notion of composition of relations is compatible with that of partial and
     total function (in ZF set theory, the latter two are defined by the former);
     associativity everywhere!
    \item NB Composition of partial functions is \emph{strict}, $f(a)=b$ and $g(b)=c$ must exist
          in order to get $g(f(a))=c$
    \item Lazy Haskell is not strict: try $(fun~x\mapsto 0)(1/0)$
 \end{itemize}

}

\frame
  {   
    \frametitle{The Category of Multimaps}\label{Ch2:Multimaps}
 \begin{itemize}[<+->]
\item Binary relation $R\subseteq A\times B$ from $A$ to $B$  can also
be seen as a \emph{multimap} $f_R$ from $A$ to $B$ which is
actually an ordinary map  $f_R : A\to \powset{B}$
\item Conversely, any $f : A\to \powset{B}$ defines a relation from $A$ to $B$
defined by $b\in f(a)$
\item To be continued

 \end{itemize}

 }
\frame
  {   
    \frametitle{Comments on script}\label{Ch1:comments}

 \begin{itemize}[<+->]
\item p. 19, Cor. 2.1.15: $in_V = in_{G_V,H_V}$, $in_E = in_{G_E,H_E}$
\item p. 29, Conv. 2.2.13: $C_V$ must be $C_E$ (twice)
\item p. 29, Rem. 2.2.15: in Haskell and type theory, composition would be
\emph{one} polymorphic function with type arguments $A,B,C$
 \end{itemize}

 }

\end{document}

\myurl{en.wikipedia.org/wiki/Ordered_pair}
 \begin{itemize}
    \item emulation halts because of the partiality of $\delta$;
    \item emulation halts because of "head left" with head at first cell;
    \item emulation halts because of reaching a halting state;
    \item the emulated TM goes on forever.
 \end{itemize}
