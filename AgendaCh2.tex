\documentclass[handout]{beamer}
%\documentclass[slides]{beamer}
% Vary the color applet  (try out your own if you like)
%\colorlet{structure}{red!20!black}
%\beamertemplateshadingbackground{yellow!20}{white}
%\usepackage{beamerthemeshadow}
%\usepackage[utf8x]{inputenc} CONFLICT!
%\usepackage[english,norsk,nynorsk]{babel}
\usepackage{tikz}
\usetikzlibrary{trees}

\usepackage[all]{xy}
\usepackage{multicol}

%\setbeamertemplate{navigation symbols}{}++++++
%\setbeamertemplate{footline}[frame number]
\usetheme{Montpellier}


\input macros

\newcommand{\To}{\Rightarrow}
\newcommand{\Trt}{\stackrel{*}{\Rightarrow}}
\newcommand{\ToG}{\Rightarrow_G}
\newcommand{\redS}{{\color{red} S}}

\newcommand{\Set}{\mathsf{Set}}

\title[INF210 presentations]{}

\begin{document}

\section{Chapter 1}
\subsection{Lecture 1}
 
\frame
  {   
    \frametitle{What is Category Theory (CT)?}\label{Ch1:What is CT}

 \begin{itemize}[<+->]
\item CT is the study of mathematical structure at a convenient abstraction level
\item Useful metaphor: ZF (Zermelo-Fraenkel, axiomatic) set theory is like an assembly language for mathematics, CT is like a higher programming language
\item The abstractions and the toolbox of CT are also very useful for CS (Computer Science)
\item We will try to give examples from (functional) programming
\item 
 
 \end{itemize}

 }

\frame
  {   
    \frametitle{Sets}\label{Ch1:sets}

 \begin{itemize}[<+->]
\item Notations: $\in$, $\emptyset$, $\cup$, $\cap$, $\backslash$, 
$\subseteq$, $\powset$, $\to$, $\circ$, $id_A : A\to A$
\item Sets are usually denoted by capital letters, with/out decoration
\item Elements of sets are usually denoted by lower case letters, with/out decorations, unless ...
\item ...  they are sets themselves and we want to stress that
\item Set builders: (explicit) $\{ \ldots \}$,  (subset) $\{x \in S \mid P(x) \}$, $P$ a predicate
 \end{itemize}

 }

\frame
  {   
    \frametitle{Cartesian Product}\label{Ch1:CartesianProduct}

 \begin{itemize}[<+->]
\item Definition: $A\times B := \set{(a,b) \mid a\in A,~b\in B}$
\item Two problems (at least) with this definition of \emph{ordered pairs}:
   \begin{itemize}[<+->]
\item Set builder without `superset' $S$
\item What in the name of Cantor is $(\_,\_)$ ?
   \end{itemize}
\item Solutions  (\myurl{en.wikipedia.org/wiki/Ordered_pair}) :
 \begin{enumerate}[<+->]
\item Set theory (Kuratowski): $(a,b)_K := \set{\set{a},\set{a,b}}$, \\
           superset $S:= \powset\powset(X\cup Y)$
\item Programming languages: Pair class in Java, tuple types in Haskell,
$(\_,\_)$ is a primitive, no worries about $S$
\item Category Theory: by a universal property (more abstract)
 \end{enumerate}
\item 1 and 2 above are implementations, 3 is `implementation independent'
 \end{itemize}

 }


\frame
  {   
    \frametitle{Universal Property of Cartesian Product}\label{Ch1:CartesianProductUP}

 \begin{itemize}[<+->]
\item Fact 1.1.1, picture and explanation on Whiteboard
\item Different implementation of $A\times B$ by $B\times A$ yields unique bijection
$(A\times B) \to (B\times A)$
\item The universal property guarantees that \emph{any} implementation
of the cartesian product of sets yields such a bijection (`implementation independent')
\item This example shows the ingredients of a category (here $\Set$):
   \begin{itemize}[<+->]
\item Objects (here: sets)
\item Arrows (here: functions)
\item Composition of (certain) arrows, identity arrows
\item Properties: associativity and identity laws
   \end{itemize}
 \end{itemize}

 }

\frame
  {   
    \frametitle{Dualization}\label{Ch1:sets}

 \begin{itemize}[<+->]
\item Dualization in CT means: reversing all arrows
\item Fact 1.1.2, picture and explanation on Whiteboard
\item Disjoint union: $A\uplus B := (A\times\set{1})\cup(A\times\set{2})$
\item Duals of projections are injections
\item Dual of $\langle f,g\rangle : C \to (A\times B)$ is $[ f,g] : (A\uplus B) \to C$ ...
\item ... and this is case distinction!
 \end{itemize}

 }


\frame
  {   
     \frametitle{More basics}\label{Ch1:setbasics}

 \begin{itemize}[<+->]
\item Functions (or maps) $A\to B$ are total unless stated otherwise, 
$A$ is the domain (or source) $B$ is the co-domain (or target)
\item We write $f: A\to B$ for $f\in (A\to B)$
\item Pairs can be generalized to $n$-tuples
\item $A^0 := \set{()}$, $A^{n+1}=A^n\times A$,
$A^* := List(A) := \cup_{n\in\nat} A^n$
\item $\powset_{fin}(A)$ is the set of finite subsets of $A$
\item Image and preimage of $f: A\to B$ for points and subsets 
\item Injective, surjective, bijective and idempotents maps
\item Inclusion map and epi-mono factorization: every map
is the (unique) composition of a surjection followed by an inclusion
 \end{itemize}

 }

\frame
  {   
    \frametitle{Some useful maps}\label{Ch1:usefulmaps}

 \begin{itemize}[<+->]
\item $del: A^* \to A^* $ removes subsequent duplicates (idempotent)
\item $sort: A^* \to A^* $, given an order on $A$, sorts (idempotent)
\item $len: A^* \to \nat $ gives the length of a list (surjective)
\item $forget: A^* \to \powset(A) $ gives the elements of a list
\item $forget^e: A^* \to \powset_{fin}(A) $ gives the elements of a list (surjective)
\item $forget$ is the composition of $forget^e$ and an inclusion
\item 
 \end{itemize}

 }

\frame
  {   
    \frametitle{Comments on script}\label{Ch1:comments}

 \begin{itemize}[<+->]
\item p. 6, last picture: $\pi'_1$ and  $\pi'_2$ interchanged
\item p. 7: notations $A.a$ and $a:A$ not good if $B=A$
 \end{itemize}

 }



\end{document}

\myurl{en.wikipedia.org/wiki/Ordered_pair}
 \begin{itemize}
    \item emulation halts because of the partiality of $\delta$;
    \item emulation halts because of "head left" with head at first cell;
    \item emulation halts because of reaching a halting state;
    \item the emulated TM goes on forever.
 \end{itemize}
