\documentclass[handout]{beamer}
%\documentclass[slides]{beamer}
% Vary the color applet  (try out your own if you like)
%\colorlet{structure}{red!20!black}
%\beamertemplateshadingbackground{yellow!20}{white}
%\usepackage{beamerthemeshadow}
%\usepackage[utf8x]{inputenc} CONFLICT!
%\usepackage[english,norsk,nynorsk]{babel}
\usepackage{tikz}
\usetikzlibrary{trees}

\usepackage[all]{xy}
\usepackage{multicol}

%\setbeamertemplate{navigation symbols}{}++++++
%\setbeamertemplate{footline}[frame number]
\usetheme{Montpellier}


\input macros

\newcommand{\To}{\Rightarrow}
\newcommand{\Trt}{\stackrel{*}{\Rightarrow}}
\newcommand{\ToG}{\Rightarrow_G}
\newcommand{\redS}{{\color{red} S}}

\title[INF210 presentations]{}

\begin{document}

\section{Chapter 1}
\subsection{Lecture 1}
 


\frame
  {
    
    \frametitle{Sets}\label{Ch1:sets}

 \begin{itemize}[<+->]
\item Notations: $\in$, $\emptyset$, $\cup$, $\cap$, $\backslash$, $\subseteq$
\item Sets are usually denoted by capital letters, with/out decoration
\item Elements of sets are usually denoted by lower case letters, with/out decorations, unless ...
\item ...  they are sets themselves and we want to stress that
\item Set building: $\{ \ldots \}$, $\{x \in S \mid P(x) \}$, $P$ a predicate
 
 \end{itemize}

 }


\end{document}

\myurl{en.wikipedia.org/wiki/Chomsky_normal_form})
 \begin{itemize}
    \item emulation halts because of the partiality of $\delta$;
    \item emulation halts because of "head left" with head at first cell;
    \item emulation halts because of reaching a halting state;
    \item the emulated TM goes on forever.
 \end{itemize}
